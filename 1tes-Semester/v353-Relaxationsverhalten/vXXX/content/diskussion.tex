\section{Diskussion}
\label{sec:Diskussion}

Es ist anzumerken, dass der erste Messwert aus \autoref{fig:daten2} nicht übernommen wurde, da er zu sehr von den anderen Messwerten anweicht,
als dass angenommen werden könnte er sei korrekt gemessen worden.
Mit den unterschiedlichen Messverfahren ergibt sich die Zeitkonstante $RC$ jeweils zu 
\begin{align*}
    RC_1 = \SI{0.705 (0.014)}{\per\second}, \\
    RC_2 = \SI{1.88 (1.54)}{\per\second}. \\
\end{align*}
Die beiden Zeitkonstanten weichen dabei nur um $37,42 \:\%$ voneinander ab.
Grund für die Abweichung ist die Messapparatur, welche keine konstante Amplitude anzeigen, oder erzeugen konnte.
Ein weiterer Grund für die Abweichung ist, dass jeder Wert manuell von einem analogen Oszilloskop abgelesen wird. 
Qualitativ ist die Theorie des Tiefpasses als teilweise verifiziert anzusehen, da der Verlauf der Spannung, wie in der Theorie beschrieben,
durch die \autoref{fig:plot_a} und \autoref{fig:Plotb} dargestellt werden.
Aus den vorhandenen Messwerten war es nicht möglich die Zeitkonstante $RC_3$ zu bestimmen. Dies macht es unmöglich Informationen aus den Messwerten zu ziehen. So ist aus der Phasenverschiebung kein deutliches Bild
abzulesen, sollte nach der Theorie ähnlich \autoref{fig:theoriepl} herauskommen.
\begin{figure}[H]
    \centering
    \includegraphics[width=0.75\textwidth]{build/Theorieplot.pdf}
    \caption{Theoriekurve nach (\ref{eqn:phasev}) mit $RC=\SI{1,6}{\per\second}$.}
    \label{fig:theoriepl}
\end{figure}

Die Theorie des RC-Kreises als Integrator wird durch \autoref{subsec:integr} ebenfalls verifiziert.