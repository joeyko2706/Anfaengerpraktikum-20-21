\section{Diskussion}
\label{sec:Diskussion}
Zunächst muss erwähnt werden, dass die Messwerte von sämtlichen Messungen der ersten Messreihe mit Druck bis $\SI{1}{\bar}$, bei weitem nicht den erwarteten Werten entsprechen.
Zudem konnte die Messung nicht bis zum angestrebten Druch von $\SI{1}{\bar}$ durchgeführt werden. Dies ist der Fall, da sich die Messapparatur ab einer Dampftemperatur von
$\SI{84}{\celsius}$ nicht mehr weiter erhitzt hat, obwohl die Heizstärke schon bei einem scheinbaren Temperaturplateau von $\SI{45}{\celsius}$ deutlich angehoben wurde.
Zusätzlich wurde bei einem weiteren Temperaturstillstand bei $\SI{76}{\celsius}$ die Kühlung des Dampfes verringert, was die Temperatur jedoch nur kurzfristig erneut
zum steigen brachte. Außerdem ist durch diesen Eingriff der Druck gesunken, was den Knick bei den Messwerten in \autoref{fig:plot1} bei $\SI{77}{\celsius}$ erklärt.

\noindent
Die Messung von $\SI{1}{\bar}$ bis $\SI{15}{\bar}$ lief ohne weitere Schwierigkeiten ab. Von den beiden Abbildungen für L ist nur \autoref{fig:plot3}, mit Addition der
Wurzel, physikalisch sinnvoll, da L am kritischen Punkt null sein muss und die Verdampfungswärme bei steigender Temperatur nicht größer werden darf.