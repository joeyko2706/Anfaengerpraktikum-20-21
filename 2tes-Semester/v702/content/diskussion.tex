\section{Diskussion}
\label{sec:Diskussion}

Zusammenfassend kann man sagen, dass die Messergebnisse sehr zufriedenstellend sind.\\
Zunächst sollte die Halbwertszeit von Vanadium berechnet werden. Diese wurde als $(245 \pm 13) \si{\second}$ bestimmt. Bei einem Literaturwert von $\SI{224,6}{\second}$
entspricht das einer Abweichung von 8,33\% \\
Die Bestimmung der Halbwertszeit von $\ce{^108 Ag}$ resultierte in einem Ergebnis von $(134 \pm 23) \si{\second}$. Dies entspricht einer Abweichung von 6,12\% vom
Literaturwert von $\SI{142,2}{\second}$.\\
Die größte Abweichung ergab sich bei der Bestimmung von $\ce{^110 Ag}$. Die gemessene und berechnete Halbwertszeit beläuft sich auf $(28,6 \pm 2,3) \si{\second}$, der
Literaturwert beträgt hingegen $\SI{142,2}{\second}$. Dies entspricht einer Abweichung von 13,99\% \\
\newline
Die wohl größte Fehlerquelle bei diesem Versuch ist die Zeit, die gebraucht wird, um die Probe aus dem Aktivierungs-Gefäß in das Zählrohr zu befördern. Da es sich um
Isotope mit sehr kurzer Halbwertszeit handelt, sind die ersten Sekunden, nach dem Rausnehmen aus der Quelle, für die Messung am wichtigsten. Genau diese Zeit konnte
bei der durchgeführten Messung nicht effektiv genutzt werden, da es zu lange dauerte, die Probe in Position zu bringen. Dies erklärt auch, warum die Abweichung bei den
Silberproben höher waren.