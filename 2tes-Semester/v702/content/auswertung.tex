\section{Auswertung}
\label{sec:Auswertung}


\subsection{Nulleffekt}
\label{sec:Nulleffekt}

Bevor die eigentliche Messung starten kann, müssen zunächst die potentiellen Einflüsse des Nulleffekts bestimmt werden. Hierfür wird eine Mesung ohne eine
Probe durchgeführt. Um statistische Schwankungen möglichst auszuklammern, wird eine vergleichsweise hohe Messzeit von $t = \SI{600}{\second}$ angesetzt.
Die Messung ergibt eine Zählrate von $N = 182$. Es werden deswegen sämtliche Messwerte der Zählrate bei der ersten Messung um $9$ und bei der zweiten
Messung um $2$ reduziert.


\subsection{Halbwertszeit von Vanadium}
\label{sec:Vanadium}

Mit der ersten Messreihe soll die Halbwertszeit von Vanadium bestimmt werden. Die hierfür gemessenen Werte sind zusammen mit ihren $\sqrt{N}$-Fehlern
in \autoref{tab:vanadium} aufgeführt. Dabei wurden die Korrekturen des Nulleffekts bereits berücksichtigt. 

\begin{table}[H]
  \centering
  \begin{tabular}{c c}
    \toprule
    Messzeit $t/\unit{\second}$ & Zählrate $N$\\
    \midrule
    30  &  167  $ \pm $  13  \\
    60  &  148  $ \pm $  12  \\
    90  &  148  $ \pm $  12  \\
    120  &  136  $ \pm $  12  \\
    150  &  128  $ \pm $  11  \\
    180  &  109  $ \pm $  10  \\
    210  &  75  $ \pm $  9  \\
    240  &  96  $ \pm $  10  \\
    270  &  85  $ \pm $  9  \\
    300  &  73  $ \pm $  9  \\
    330  &  70  $ \pm $  8  \\
    360  &  58  $ \pm $  8  \\
    390  &  72  $ \pm $  8  \\
    420  &  52  $ \pm $  7  \\
    450  &  51  $ \pm $  7  \\
    480  &  56  $ \pm $  7  \\
    510  &  40  $ \pm $  6  \\
    540  &  41  $ \pm $  6  \\
    570  &  34  $ \pm $  6  \\
    600  &  40  $ \pm $  6  \\
    630  &  29  $ \pm $  5  \\
    660  &  42  $ \pm $  6  \\
    690  &  19  $ \pm $  4  \\
    720  &  15  $ \pm $  4  \\
    750  &  15  $ \pm $  4  \\
    780  &  22  $ \pm $  5  \\
    810  &  13  $ \pm $  4  \\
    840  &  28  $ \pm $  5  \\
    870  &  20  $ \pm $  4  \\
    900  &  13  $ \pm $  4  \\
    \bottomrule
  \end{tabular}
  \caption{Messwerte der Zählrate für Vanadium.}
  \label{tab:vanadium}
\end{table}

Zur Bestimmung der Halbwertszeit wird die Zälrate halblogarithmisch gegen die Zeit aufgetragen und eine Ausgleichsgerade eingezeichnet.
Dies ist in \autoref{fig:vanadium} dargestellt.

\begin{figure}[H]
  \centering
  \includegraphics{plot1.pdf}
  \caption{Messwerte und lineare Regression für Vanadium.}
  \label{fig:vanadium}
\end{figure}

Mit Hilfe von Scipy wurde eine Ausgleichsrechnung durchgeführt. Diese liefert zusammen mit Gleichung \eqref{eqn:Zerfallsgesetz} die Werte

\begin{align*}
  \lambda &= (-2,830 \pm 0,151) \cdot 10^{-3} \frac{1}{\si{\second}} \\
  ln(N_0) &= 5,193 \pm 0,0804 .
\end{align*}

Es kann anschließend über Gleichung \eqref{eqn:T} die Halbwertszeit $T$ zu

\begin{align*}
  T &= (245 \pm 13) \si{\second}
\end{align*}

bestimmt werden.


\subsection{Halbwertszeit von Silber}
\label{sec:Silber}

Im zweiten Teil der Messung soll nun die Halbwertszeit von Silber bestimmt werden. Die hierfür gemessenen Werte sind zusammen mit ihren $\sqrt{N}$-Fehlern
in \autoref{tab:silber} aufgeführt. Es sind erneut die Korrekturen bezüglich des Nulleffekts bereits berücksichtigt.

\begin{table}[H]
  \centering
  \begin{tabular}{c c}
    \toprule
    Messzeit $t/\unit{\second}$ & Zählrate $N$\\
    \midrule
    8  &  134  $ \pm $  12  \\
    16  &  95  $ \pm $  10  \\
    24  &  118  $ \pm $  11  \\
    32  &  91  $ \pm $  10  \\
    40  &  75  $ \pm $  9  \\
    48  &  67  $ \pm $  8  \\
    56  &  58  $ \pm $  8  \\
    64  &  48  $ \pm $  7  \\
    72  &  32  $ \pm $  6  \\
    80  &  28  $ \pm $  5  \\
    88  &  21  $ \pm $  5  \\
    96  &  33  $ \pm $  6  \\
    104  &  27  $ \pm $  5  \\
    112  &  29  $ \pm $  5  \\
    120  &  19  $ \pm $  4  \\
    128  &  18  $ \pm $  4  \\
    136  &  17  $ \pm $  4  \\
    144  &  25  $ \pm $  5  \\
    152  &  13  $ \pm $  4  \\
    160  &  20  $ \pm $  4  \\
    168  &  11  $ \pm $  3  \\
    176  &  10  $ \pm $  3  \\
    184  &  15  $ \pm $  4  \\
    192  &  14  $ \pm $  4  \\
    200  &  5  $ \pm $  2  \\
    208  &  11  $ \pm $  3  \\
    216  &  20  $ \pm $  4  \\
    224  &  12  $ \pm $  3  \\
    232  &  6  $ \pm $  2  \\
    240  &  12  $ \pm $  3  \\
    248  &  19  $ \pm $  4  \\
    256  &  5  $ \pm $  2  \\
    264  &  15  $ \pm $  4  \\
    272  &  4  $ \pm $  2  \\
    280  &  3  $ \pm $  2  \\
    288  &  4  $ \pm $  2  \\
    296  &  12  $ \pm $  3  \\
    304  &  9  $ \pm $  3  \\
    312  &  5  $ \pm $  2  \\
    320  &  6  $ \pm $  2  \\
    328  &  10  $ \pm $  3  \\
    336  &  5  $ \pm $  2  \\
    344  &  8  $ \pm $  3  \\
    352  &  12  $ \pm $  3  \\
    360  &  5  $ \pm $  2  \\
    368  &  5  $ \pm $  2  \\
    376  &  2  $ \pm $  1  \\
    384  &  1  $ \pm $  1  \\
    392  &  9  $ \pm $  3  \\
    400  &  8  $ \pm $  3  \\
    408  &  6  $ \pm $  2  \\
    416  &  7  $ \pm $  3  \\
    424  &  4  $ \pm $  2  \\
    \bottomrule
  \end{tabular}
  \caption{Messwerte der Zählrate für Silber.}
  \label{tab:silber}
\end{table}

Da Silber natürlicherweise zu 52,3\% in Form des Isotops $\ce{^108 Ag}$ und zu 47,7\% in Form von $\ce{^110 Ag}$ vorkommt,
muss für beide Isotope seperat die Halbwertszeit bestimmt werden. Um die beiden Zerfälle zu trennen, wurden die Messwerte an der Stelle $t = \SI{96}{\second}$
in zwei Bereiche aufgeteilt. Auf der linken Seite sind beide Zerfälle vorhanden, auf der rechten dann nur noch der langlebige Zerfall von $\ce{^108 Ag}$.
Die Messwerte sind graphisch in \autoref{fig:silber} aufgetragen.

\begin{figure}[H]
  \centering
  \includegraphics{plot2.pdf}
  \caption{Messwerte und lineare Regression für Silber.}
  \label{fig:silber}
\end{figure}



\subsubsection{Halbwertszeit von $\ce{^108 Ag}$}

Zur Bestimmung der Halbwertszeit von $\ce{^108 Ag}$ kann, wie im ersten Teil des Versuches, eine einfache Ausgleichsrechnung durgeführt werden. Zusammen mit
Gleichung \eqref{eqn:Zerfallsgesetz} kann dann $\lambda_{108}$ bestimmt werden. Die Berechnungen liefern die Werte
\begin{align*}
  \lambda_{108} &= (-5,190 \pm 0,883) \cdot 10^{-3} \frac{1}{\si{\second}} \\
  ln(N_0)_{108} &=  3,539 \pm 0,248.
\end{align*}

Anschließend kann über Gleichung \eqref{eqn:T} die Halbwertszeit $T$ zu

\begin{align*}
  T_{108} &= (134 \pm 23) \si{\second}
\end{align*}

bestimmt werden.


\subsubsection{Halbwertszeit von $\ce{^110 Ag}$}

Da der Zerfall von $\ce{^110 Ag}$ schneller passiert, als der von $\ce{^108 Ag}$, muss die theoretische Zählrate des langsamen Zerfalls von der gesamten Zählrate
subtrahiert werden.\\
Die resultierende Zählrate kann so erneut über eine Ausgleichsrechnung und Gleichung \eqref{eqn:lnDelta} bestimmt werden.



Die korrigierten Werte für die lineare Regression ergeben sich aus
\begin{align*}
  N_{\symup{110}}(t) &= N_{\symup{Ges}}(t) - N_{\symup{108}}(t) \\
                      &= N_{\symup{Ges}}(t) - e^{-mt + b}
\end{align*}
Aus der Gaußschen Fehlerforpflanzung ergibt sich
\begin{align*}
  \Delta N_{\symup{110}}(t) &= \sqrt{
    \left(\frac{\partial N_{\symup{110}}}{\partial N_{\symup{Ges}}}\right)^2 (\Delta N_{\symup{Ges}})^2 +
    \left(\frac{\partial N_{\symup{110}}}{\partial m}\right)^2 (\Delta m)^2 + 
    \left(\frac{\partial N_{\symup{110}}}{\partial b}\right)^2 (\Delta b)^2} \\
  \Delta N_{\symup{110}}(t) &= \sqrt{
    (\Delta N_{\symup{Ges}})^2 +
    \left(te^{-mt+b}\right)^2 (\Delta m)^2 +
    \left(e^{-mt+b}\right)^2 (\Delta b)^2
  }\,. \\
\end{align*}




Die Berechnungen resultieren in den Werten
\begin{align*}
  \lambda_{110} &= (-24,2487 \pm 1,982) \cdot 10^{-3} \frac{1}{\si{\second}} \\
  ln(N_0)_{110} &=  5,153 \pm 0,108.
\end{align*}

Es kann anschließend über Gleichung \eqref{eqn:T} die Halbwertszeit $T$ zu

\begin{align*}
  T_{110} &= (28,6 \pm 2,3) \si{\second}
\end{align*}

bestimmt werden.
