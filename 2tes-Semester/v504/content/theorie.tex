\section{Zielsetzung}
Ziel dieses Versuches ist die Untersuchung der Temperaturabhängigkeit des glühelektrischen Effektes und konkret die Bestimmung der Austrittsarbeit für Elektronen im
hierfür verwendeten Material Wolfram.

\section{Theorie}
\label{sec:Theorie}

Die meisten Metalle sind kristaline Festkörper, die eine hervorragende elektrische Leitfähigkeit vorweisen. Der Grund dafür ist eben diese kristaline Struktur, wegen
der es im Metall freigesetzte Elektronen gibt, die zu keinem Atom zugehörig sind und als Leitungselektronen bezeichnet werden. Im Metallinneren gibt es somit ein
einheitliches Potential, das sich jedoch um einen festen Betrag $\phi$ von Außenraum unterscheidet, weswegen keine Kräfte auf die Elektronen innerhalb wirken.
\newline
Gegen dieses Potential muss ein Elektron, dass das Metall verlassen soll, also ankommen. Die Arbeit, die es für diesen Übergang leisten muss wird als Austrittsarbeit
bezeichnet. Wie wahrscheinlich es ist, dass ein Elektron im Metall genug Energie hat, um diese Austrittsarbeit zu erbringen, ist durch die Fermi-Diracsche
Verteilungs-Funktion 
\begin{equation}
    \label{eqn:Fermi_Dirac}
    f(E) = \frac{1}{\mathrm{exp}\left( \frac{\zeta - E}{kT} \right) + 1}
\end{equation}
gegeben. Da für den konkreten Versuch Wolfram als Kathodenmaterial verwendet wurde, kann diese Funktion vereinfacht als
\begin{equation}
    \label{eqn:Fermi_Dirac}
    f(E) \approx \mathrm{exp}\left( \frac{\zeta - E}{kT} \right)
\end{equation}
angenommen werden.

%\subsubsection{Sättigungsstrom}

Wie viele Elektronen pro Zeit und pro Fläche abhängig von der Temperatur aus einer festen Metalloberfläche austreten, ist durch die Richardson-Gleichung 
\begin{equation}
    j_\text{S}(T) = 4 \symup{\pi}\frac{e m_0 k^2}{h^3}T^2 \cdot \symup{e}^{-\frac{e\phi}{kT}}
    \label{eqn:J_S}
\end{equation}
gegeben. Dabei beschreibt $j_\text{S}(T)$ die Sättigungsstromdichte, $h$ das Plancksche Wirkungsquantum, $m_0$ die Ruhemasse eines Elektrons, $e_0$ die Elementarladung,
$T$ die Temperatur und $\phi$ die Potentialdifferenz zwischen Außenraum und Metallinnerem.

%\subsubsection{Raumladungsgebiet}

Da die Elektronen im verwendeten Versuchsaufbau eine beschleunigte Bewegung von Kathode zu Anode durchführen, verliert das Ohmsche Gesetz seine Gültigkeit. Die Stromdichte $j$,
die laut Kontinuitätsbedingung als $j = - \rho v$ gegeben ist, wobei $\rho$ die Raumladungsdichte und $v$ die Geschwindigkeit der Elektronen sind, muss laut dieser allerdings
konstant sein. Daraus und aus der nicht-konstanten Geschwindigkeit folgt, dass $\rho$ abhängig vom Ort ist und zur Anode hin, also mit steigender Geschwindigkeit, sinkt.
\newline
Da das Ohmsche Gesetz hier nicht gültig ist, wächst die Stromdichte also nicht proportional zur Anodenspannung. Das tatsächliche Wachstum der Stromdichte kann stattdessen
durch das Langmuir-Schottkysche Raumladungsgesetz 
\begin{equation}
    j = \frac{4}{9}\varepsilon_0 \sqrt{\frac{2e}{m_0}}\frac{V^{3/2}}{a^2}
    \label{eqn:Raumladung}
\end{equation}
beschrieben werden, wobei $V$ die Anodenspannung und $a$ der Abstand von Kathode zu Anode sind.

%\subsubsection{Anlaufstrom}
Aus Gleichung \eqref{eqn:Raumladung} folgt, dass die Stromdichte $j$ für eine Anodenspannung $V = 0$ ebenfalls 0 sein müsste. Dies entspricht allerdings nicht der Realität,
da auch ohne Anodenspannung und sogar mit einer kleinen entgegengesetzten Spannung, die die austretenden Elektronen verlangsamt, eine Stromdichte $j \neq 0$ vorhanden ist.
Der Grund dafür ist, dass die austreten Elektronen eine Energie besitzen, die teilweise größer ist, als die, die zum Austritt benötigt wird. Ist dies der Fall, besitzen diese
Elektronen nach dem Austritt eine kinetische Energie, mit der sie sich, trotz einer entgegengesetzte Spannung, hin zur Anode bewegen können. Ein Ausdruck für diese
Stromdichte für Anodenspannungen $V \leq 0$ ist durch
\begin{equation}
    \label{eqn:j_Anlauf}
    j(V) = j_0 \mathrm{exp}\left(-\frac{e\phi_\text{A} + e \cdot V}{kT} \right) = const \cdot \mathrm{e}^{-\frac{e \cdot V}{kT}}
\end{equation}
gegeben.

%\subsection{Temperaturberechnung der Kathode}
\label{subsec:Temperatur_Kathode}
Aus dem Stefan-Boltzmannschen Gesetz folgt als Ausdruck für die Strahlungsleistung
\begin{equation*}
    N_\text{W} = f \eta \sigma T^4,
\end{equation*}
wobei $\sigma = \SI{5,7}{\watt\per\centi\metre\squared\kelvin^4}$ die Stefan-Boltzmannsche Strahlungskonstante, $f$ die emittierende Kathodenoberfläche und $\eta = 0,28$ der
Emissionsgrad der Oberfläche sind. Ist die Wärmeleistung $N_\text{WL}$ der Kathode bekannt ergibt sich mit der zugeführten Leistung aus $I_\text{f}$ und $U_\text{f}$
\begin{equation}
    \label{eqn:Leistung}
    I_\text{f} U_\text{f} = f \eta \sigma T^4 - N_\text{WL}.
\end{equation}
Mit diesem Ausdruck lässt sich dann die Temperatur $T$ der Kathode bestimmen.
\cite{Anleitung504}

% -z68 Wert bei SI ist falsch
% -subsections