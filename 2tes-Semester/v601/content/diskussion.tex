\section{Diskussion}
\label{sec:Diskussion}

Zusammenfassen kann gesagt werden, dass sämtliche Messungen erfolgreich verlaufen sind und sich die Ergebnisse größtenteils mit den Erwartungen decken.
\newline
Die Abstände der Maxima bei den Franck-Hertz-Kurven wurden für die Kurve bei $T = \SI{443,16}{\kelvin}$ als $\SI{4,84 \pm 0,12}{\volt}$
und für die Kurve bei $\SI{463,16}{\kelvin}$ als $\SI{4,88 \pm 0,15}{\volt}$ gemessen. Die Abweichung zum Literaturwert von $\SI{4,9}{\volt}$
beträgt somit für die Kurve bei $T = \SI{443,16}{\kelvin}$ 1,22\% und für die bei $T = \SI{463,16}{\kelvin}$ 0,41\%.
\newline \newline
Bei allen Messung ist anzumerken, dass die Messung der Temperatur wohl sehr ungenau ausgefallen ist. Nach Erhöhungen der Heizleistung,
zeigte der Temperaturmesser zum Beispiel teils niedrigere Temperaturen an. Darüber hinaus schwankte die gemessene Temperatur auch bei konstanter Heizleistung teils stark.
Ursache hierfür könnten Ungenauigkeiten der verwendeten Apparaturen sein, die auf Bauweise und Alter dieser zurückzuführen sind. Außerdem wurden sämtliche Messungen
zunächst bei geöffnetem Fenster durchgeführt. Dieser Umstand wurde allerdings wärend der Versuchsdurchführung korrigiert, was weitere Inkonsequenzen mit sich gebracht
haben könnte.
\newline
Des Weiteren könnten durch das Auswerten der vom XY-Schreiber gefertigten Kurven weitere Ungenauigkeiten aufgetreten sein. Vor allem das nachträgliche Skalieren der
Kurven erwies sich als schwierig, da der XY-Schreiber nach dem Zurückregeln der Spannung teilweise weit vom ursprünglichen Startpunkt entfernt war. Es mussten somit
bei der Auswertung Mittelwerte der Abstände in der Skalierung gebildet werden, was zur Folge hatte, dass die angegeben Werte für Maxima teilweise augenscheinlich
nicht mit der Skalierung übereinstimmen.