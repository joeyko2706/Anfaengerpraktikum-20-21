\section{Auswertung}
\label{sec:Auswertung}
Es werden die drei Stoffe Neodym, Gadolinium und Dysprosium untersucht. Um die theoretischen Werte der magnetischen Suszeptibilität berechnen zu können, müssen zunächst die Quantenzahlen bestimmt werden.
Dafür muss auf den Aufbau der drei Proben eingegangen werden. Wie Werte sind in \autoref{tab:quant} zusammengefasst.
Alle Stoffe sind Metalle der Seltenen Erden und haben damit eine komplette 5p-Schale und 2s-Elektronen, die aber für den Paramagnetismus unwichtig sind, da die Spins und Bahndrehimpulse abgesättigt sind
und der resultierende Drehimpuls verschwindet. Da die Seltenen Erd Metalle alle ein 4f-Elektron mehr haben, als ihre Vorgänger, folgen die Anzahl der 4f-Elektronen jeweils für die drei Stoffe.
Somit hat Neodym 4 Elektronen, Gadolinium hat 8  und Dysprosium hat dementsprechend 10 Elektronen auf der 4f-Schale.
Wenn ein Atom Ionisiert wird, verliert es ein 6s- und ein 4f-Elektron.
Gemäß der Hund'schen Regeln [hier Referenz einfügen] werden die Quantenzahlen berechnet.
Der Landé-Faktor wird nach [Gleichung referenzieren] berechnet.

\begin{table}%Die Berechnung von S ist wahrscheinlich falsch und ich habe keine Ahnung, wie man L berechnet.
  \centering
  \caption{Bestimmung der Quantenzahlen.}
  \label{tab:quant}
  \begin{tabular}{l c c c c c c}
      \toprule
       & 
      neutral & 
      \multicolumn{4}{c}{ionisiert} \\
      \cmidrule(lr){2-2} \cmidrule(lr){3-7}
      Element &
      4f & 
      4f &
      $S$ & 
      $L$ & 
      $J$ &
      $g_J$ \\
      \midrule
      Nd &  4 & 3 & $3\cdot\sfrac{1}{2}=\sfrac{3}{2}$ & 6 & $\sfrac{9}{2}$  & 0.727 \\
      Gd &  8 & 7 & $7\cdot\sfrac{1}{2}=\sfrac{7}{2}$ & 0 & $\sfrac{7}{2}$ & 2.000 \\
      Dy & 10 & 9 & $9\cdot\sfrac{1}{2}=\sfrac{9}{2}$ & 5 & $\sfrac{45}{2}$ & 1.495 \\
      \bottomrule
  \end{tabular}
\end{table}

In \autoref{tab:abmProben} sind die Abmaße der drei in dem Versuch verwendeten Proben zu entnehmen.
\begin{table}[H]
  \centering
  \caption{Abmaße der drei Proben.}
  \label{tab:abmProben}
  \begin{tabular}{c c c c}
    \toprule
    Messgröße & Neodym & Gadolinium & Dysprosium \\
    \midrule
    Masse $m$ & $\SI{18.48}{\gram}$ & $\SI{14.08}{\gram}$ & $\SI{15.10}{\gram}$ \\
    Länge $l$ & $\SI{18.2}{\centi\meter}$ & $\SI{17.1}{\centi\meter}$ & $\SI{16.6}{\centi\meter}$ \\ %Bei Dysprosium muss noch angegeben werden, dass die Luftblase bereits rausgerechnet wurde (mit Werten?)!
    Durchmesser $d$ & $\SI{0.8}{\centi\meter}$ & $\SI{0.7}{\centi\meter}$ & $\SI{0.7}{\centi\meter}$ \\
    \bottomrule
  \end{tabular}
\end{table}

In \autoref{tab:NdMessw} sind die Messwerte zu den drei Durchläufen der Messung der Widerstände und Spannungen der Brückenschaltung bei Neodymfüllung zu entnehmen.
\begin{table}[H]
  \centering
  \caption{Messwerte der Widerstände und Spannungen der Brückenschaltung bei $\SI{35}{\kilo\hertz}$ und $\SI{60}{\milli\volt}$ bei Neodym.}
  \label{tab:NdMessw}
  \begin{tabular}{c| c c c c}
    \toprule
    Durchlauf & $R_3 \,/\, \si{\ohm}$ & $R_4 \,/\, \si{\ohm}$ & $\upDelta R \,/\, \si{\ohm}$ & $U_{\text{Br}} \,/\, \si{\milli\volt}$ \\%Welche Brückenspannung ist hier gefragt? Ich bin von der Ausgegangen, die am Voltmeter angezeigt wird, wenn die Probe eingesteckt wird
    \midrule
    Erster & 0.925 & 4.075 & 0.175 & 49 \\
    Zweiter & 0.925 & 4.075 & 0.200 & 49 \\
    Dritter & 0.925 & 4.075 & 0.190 & 49 \\
    Mittelwerte & 0.925 & 4.075 & 0.188 & 49\\
    \bottomrule
  \end{tabular}
\end{table}


In \autoref{tab:GdMessw} sind die Messwerte zu den drei Durchläufen der Messung der Brückenwiderstände und -Spannungen bei Gadoliniumfüllung zu entnehmen.
\begin{table}[H]
  \centering
  \caption{Messwerte der Widerstände und Spannungen der Brückenschaltung bei $\SI{35}{\kilo\hertz}$ und $\SI{60}{\milli\volt}$ bei Gadolinium.}
  \label{tab:GdMessw}
  \begin{tabular}{c| c c c c}
    \toprule
    Durchlauf & $R_3 \,/\, \si{\ohm}$ & $R_4 \,/\, \si{\ohm}$ & $\upDelta R \,/\, \si{\ohm}$ & $U_{\text{Br}} \,/\, \si{\milli\volt}$ \\
    \midrule
    Erster & 0.925 & 4.075 & 0.90 & 67.0 \\
    Zweiter & 0.925 & 4.075 & 0.90 & 69.0 \\
    Dritter & 0.925 & 4.075 & 0.80 & 69.0 \\
    Mittelwerte & 0.925 & 4.075 & 0.87 & 68.3\\
    \bottomrule
  \end{tabular}
\end{table}


In \autoref{tab:DyMessw} sind die Messwerte zu den drei Durchläufen der Messung der Widerstände und Spannungen der Brückenschaltung bei Dysprosiumfüllung zu entnehmen. Es ist darauf zu achten,
dass der Lufteinschluss in der Probe, die eine Länge von $\SI{0.8}{\centi\meter}$ hat, bereits aus der Länge der Probe herausgerechnet wurde.
\begin{table}[H]
  \centering
  \caption{Messwerte der Widerstände und Spannungen der Brückenschaltung bei $\SI{35}{\kilo\hertz}$ und $\SI{60}{\milli\volt}$ bei Dysprosium.}
  \label{tab:DyMessw}
  \begin{tabular}{c| c c c c}
    \toprule
    Durchlauf & $R_3 \,/\, \si{\ohm}$ & $R_4 \,/\, \si{\ohm}$ & $\upDelta R \,/\, \si{\ohm}$ & $U_{\text{Br}} \,/\, \si{\milli\volt}$ \\
    \midrule
    Erster & 0.925 & 4.075 & 1.675 & 30.75 \\
    Zweiter & 0.925 & 4.075 & 1.675 & 30.75 \\
    Dritter & 0.925 & 4.075 & 1.675 & 30.75 \\
    Mittelwerte & 0.925 & 4.075 & 1.675 & 30.75 \\
    \bottomrule
  \end{tabular}
\end{table}

In \autoref{tab:alleMessw} sind alle Mittelwerte der oben aufgeführten Messdaten der Brückenwiderstände und -Spannungen noch einmal aufgeführt. Zudem sind die Messwerte für einen Durchlauf ohne Füllung zu entnehmen.
Es ist darauf zu achten, dass für die Messung der Dysprosiumfüllung das Millivoltmeter umskaliert wurde, weshalb Die Brückenschaltung noch ein zweites Mal ohne eine Probe vermessen werden musste.
\begin{table}[H]
  \centering
  \caption{Mittlere Messwerte der Widerstände und Spannungen der Brückenschaltung bei $\SI{35}{\kilo\hertz}$ und $\SI{60}{\milli\volt}$ in Abhängigkeit der Spulenfüllung.}
  \label{tab:alleMessw}
  \begin{tabular}{c| c c c c}
    \toprule
    Füllung & $R_3 \,/\, \si{\ohm}$ & $R_4 \,/\, \si{\ohm}$ & $\upDelta R \,/\, \si{\ohm}$ & $U_{\text{Br}} \,/\, \si{\milli\volt}$ \\
    \midrule
    ohne Probe & 0.925 & 4.075 & 0.900 & 67.00 \\
    Neodym & 0.925 & 4.075 & 0.188 & 49.00\\ 
    Gadolinium & 0.925 & 4.075 & 0.870 & 68.30\\ \hline
    ohne Probe & 0.925 & 4.075 & 0.900 & 13.50 \\
    Dysprosium & 0.925 & 4.075 & 1.675 & 30.75 \\
    \bottomrule
  \end{tabular}
\end{table}