\section{Auswertung}
\label{sec:Auswertung}

In \autoref{tab:abmProben} sind die Abmaße der drei in dem Versuch verwendeten Proben zu entnehmen.
\begin{table}[H]
  \centering
  \caption{Abmaße der drei Proben.}
  \label{tab:abmProben}
  \begin{tabular}{c c c c}
    \toprule
    Messgröße & Neodym & Gadolinium & Dysprosium \\
    \midrule
    Masse $m$ & $\SI{18.48}{\gram}$ & $\SI{14.08}{\gram}$ & $\SI{15.10}{\gram}$ \\
    Länge $l$ & $\SI{18.2}{\centi\meter}$ & $\SI{17.1}{\centi\meter}$ & $\SI{16.6}{\centi\meter}$ \\ %Bei Dysprosium muss noch angegeben werden, dass die Luftblase bereits rausgerechnet wurde (mit Werten?)!
    Durchmesser $d$ & $\SI{0.8}{\centi\meter}$ & $\SI{0.7}{\centi\meter}$ & $\SI{0.7}{\centi\meter}$ \\
    \bottomrule
  \end{tabular}
\end{table}

In \autoref{tab:NdMessw} sind die Messwerte zu den drei Durchläufen der Messung der Widerstände und Spannungen der Brückenschaltung bei Neodymfüllung zu entnehmen.
\begin{table}[H]
  \centering
  \caption{Messwerte der Widerstände und Spannungen der Brückenschaltung bei $\SI{35}{\kilo\hertz}$ und $\SI{60}{\milli\volt}$ bei Neodym.}
  \label{tab:NdMessw}
  \begin{tabular}{c| c c c c}
    \toprule
    Durchlauf & $R_3 \,/\, \si{\ohm}$ & $R_4 \,/\, \si{\ohm}$ & $\upDelta R \,/\, \si{\ohm}$ & $U_{\text{Br}} \,/\, \si{\milli\volt}$ \\%Welche Brückenspannung ist hier gefragt? Ich bin von der Ausgegangen, die am Voltmeter angezeigt wird, wenn die Probe eingesteckt wird
    \midrule
    Erster & 0.925 & 4.075 & 0.175 & 49 \\
    Zweiter & 0.925 & 4.075 & 0.200 & 49 \\
    Dritter & 0.925 & 4.075 & 0.190 & 49 \\
    Mittelwerte & 0.925 & 4.075 & 0.188 & 49\\
    \bottomrule
  \end{tabular}
\end{table}


In \autoref{tab:GdMessw} sind die Messwerte zu den drei Durchläufen der Messung der Brückenwiderstände und -Spannungen bei Gadoliniumfüllung zu entnehmen.
\begin{table}[H]
  \centering
  \caption{Messwerte der Widerstände und Spannungen der Brückenschaltung bei $\SI{35}{\kilo\hertz}$ und $\SI{60}{\milli\volt}$ bei Gadolinium.}
  \label{tab:GdMessw}
  \begin{tabular}{c| c c c c}
    \toprule
    Durchlauf & $R_3 \,/\, \si{\ohm}$ & $R_4 \,/\, \si{\ohm}$ & $\upDelta R \,/\, \si{\ohm}$ & $U_{\text{Br}} \,/\, \si{\milli\volt}$ \\
    \midrule
    Erster & 0.925 & 4.075 & 0.90 & 67.0 \\
    Zweiter & 0.925 & 4.075 & 0.90 & 69.0 \\
    Dritter & 0.925 & 4.075 & 0.80 & 69.0 \\
    Mittelwerte & 0.925 & 4.075 & 0.87 & 68.3\\
    \bottomrule
  \end{tabular}
\end{table}


In \autoref{tab:DyMessw} sind die Messwerte zu den drei Durchläufen der Messung der Widerstände und Spannungen der Brückenschaltung bei Dysprosiumfüllung zu entnehmen. Es ist darauf zu achten,
dass der Lufteinschluss in der Probe, die eine Länge von $\SI{0.8}{\centi\meter}$ hat, bereits aus der Länge der Probe herausgerechnet wurde.
\begin{table}[H]
  \centering
  \caption{Messwerte der Widerstände und Spannungen der Brückenschaltung bei $\SI{35}{\kilo\hertz}$ und $\SI{60}{\milli\volt}$ bei Dysprosium.}
  \label{tab:DyMessw}
  \begin{tabular}{c| c c c c}
    \toprule
    Durchlauf & $R_3 \,/\, \si{\ohm}$ & $R_4 \,/\, \si{\ohm}$ & $\upDelta R \,/\, \si{\ohm}$ & $U_{\text{Br}} \,/\, \si{\milli\volt}$ \\
    \midrule
    Erster & 0.925 & 4.075 & 1.675 & 30.75 \\
    Zweiter & 0.925 & 4.075 & 1.675 & 30.75 \\
    Dritter & 0.925 & 4.075 & 1.675 & 30.75 \\
    Mittelwerte & 0.925 & 4.075 & 1.675 & 30.75 \\
    \bottomrule
  \end{tabular}
\end{table}

In \autoref{tab:alleMessw} sind alle Mittelwerte der oben aufgeführten Messdaten der Brückenwiderstände und -Spannungen noch einmal aufgeführt. Zudem sind die Messwerte für einen Durchlauf ohne Füllung zu entnehmen.
Es ist darauf zu achten, dass für die Messung der Dysprosiumfüllung das Millivoltmeter umskaliert wurde, weshalb Die Brückenschaltung noch ein zweites Mal ohne eine Probe vermessen werden musste.
\begin{table}[H]
  \centering
  \caption{Mittlere Messwerte der Widerstände und Spannungen der Brückenschaltung bei $\SI{35}{\kilo\hertz}$ und $\SI{60}{\milli\volt}$ in Abhängigkeit der Spulenfüllung.}
  \label{tab:alleMessw}
  \begin{tabular}{c| c c c c}
    \toprule
    Füllung & $R_3 \,/\, \si{\ohm}$ & $R_4 \,/\, \si{\ohm}$ & $\upDelta R \,/\, \si{\ohm}$ & $U_{\text{Br}} \,/\, \si{\milli\volt}$ \\
    \midrule
    ohne Probe & 0.925 & 4.075 & 0.900 & 67.00 \\
    Neodym & 0.925 & 4.075 & 0.188 & 49.00\\ 
    Gadolinium & 0.925 & 4.075 & 0.870 & 68.30\\ \hline
    ohne Probe & 0.925 & 4.075 & 0.900 & 13.50 \\
    Dysprosium & 0.925 & 4.075 & 1.675 & 30.75 \\
    \bottomrule
  \end{tabular}
\end{table}