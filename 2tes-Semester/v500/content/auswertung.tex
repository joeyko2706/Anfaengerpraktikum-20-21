\section{Auswertung}
\label{sec:Auswertung}

Es werden jeweils für die fünf untersuchten Farben bzw. Wellenlängen die eingestellten Bremsspannungen $U_\text{B}$ gegen die Wurzel des gemessenen Photostromes $\sqrt{I_\text{P}}$
geplottet. Die Darstellung der Messergebnisse für die rote Spektrallinie sind in \autoref{fig:rot}, die für die orangene in \autoref{fig:orange},
die für die grüne in \autoref{fig:gruen}, die für die blaue in \autoref{fig:blau} und die für die violette in \autoref{fig:violett} zu finden.
Für jede Messreihe wurde zusätzlich eine lineare Regression berechnet und eingezeichnet.

\begin{figure}[H]
  \centering
  \includegraphics[width=\textwidth]{rot.pdf}
  \caption{Wurzel des Photostroms in Relation zur Bremsspannung für die rote Spektrallinie.}
  \label{fig:rot}
\end{figure}

\begin{figure}[H]
  \centering
  \includegraphics[width=\textwidth]{orange.pdf}
  \caption{Wurzel des Photostroms in Relation zur Bremsspannung für die orangene Spektrallinie.}
  \label{fig:orange}
\end{figure}

\begin{figure}[H]
  \centering
  \includegraphics[width=\textwidth]{gruen.pdf}
  \caption{Wurzel des Photostroms in Relation zur Bremsspannung für die grüne Spektrallinie.}
  \label{fig:gruen}
\end{figure}

\begin{figure}[H]
  \centering
  \includegraphics[width=\textwidth]{blau.pdf}
  \caption{Wurzel des Photostroms in Relation zur Bremsspannung für die blaue Spektrallinie.}
  \label{fig:blau}
\end{figure}

\begin{figure}[H]
  \centering
  \includegraphics[width=\textwidth]{violett.pdf}
  \caption{Wurzel des Photostroms in Relation zur Bremsspannung für die violette Spektrallinie.}
  \label{fig:violett}
\end{figure}

\noindent
Mit Hilfe der linearen Regressionen, die der Form $y=a \cdot x+b$ entsprechen, lassen sich nun die verschiedenen Gegenspannungen $U_\text{G}$ bestimmen. Es werden hiefür
$y=\sqrt{I_\text{P}}$, $x=U_\text{B}$ und $U_\text{G}=-\frac{b}{a}$ verwendet. Die jeweiligen Werte für $a$ und $b$ und die aus ihnen resultierenden Spannungen
$U_\text{G}$ sind mit den zugehörigen Farben in \autoref{tab:ug} aufgeführt.

\begin{table}[H]
  \centering
  \begin{tabular}{c c c c}
      \toprule
      Farbe & $a$ & $b$ & $U_\text{G} / \si{\volt}$\\
      \midrule
      Rot &  -1,77 $\pm$ 0,11 & 0,71 $\pm$ 0,03 & 0,40 $\pm$ 0,03 \\
      Orange &  -2,91 $\pm$ 0,23 & 1,62 $\pm$ 0,07 & 0,56 $\pm$ 0,05 \\
      Grün & -3,35 $\pm$ 0,12 & 2,31 $\pm$ 0,05 & 0,69 $\pm$ 0,03 \\
      Blau &  -2,78 $\pm$ 0,08 & 3,26 $\pm$ 0,05 & 1,17 $\pm$ 0,04 \\
      Violett &  -2,24 $\pm$ 0,07 & 3,08 $\pm$ 0,05 & 1,38 $\pm$ 0,05 \\
      \bottomrule
  \end{tabular}
  \caption{Werte der linearen Regression.}
  \label{tab:ug}
\end{table}

\noindent
Die berechneten Gegenspannungen werden nun gegen die jeweilige Frequenz des Lichtes geplottet und erneut eine lineare Regression durchgeführt. Dies ist in \autoref{fig:gegen} dargestellt.

\begin{figure}[H]
  \centering
  \includegraphics[width=\textwidth]{gegenspannung.pdf}
  \caption{Gegenspannungen mit jeweils zugehöriger Frequenz.}
  \label{fig:gegen}
\end{figure}

\noindent
Aus der linearen Regression folgt mit Gleichung \eqref{eqn:energieElektr} $a=\frac{h}{e_0}$ und $b=A_\text{K}$. Dies liefert die Werte
\begin{align*}
  \frac{h}{e_0} &= (4,66 \pm 0,58) \cdot 10^{-15} \si{\volt\second}, \\
  A_\text{K} &= \SI{1,81 \pm 0,33}{\electronvolt}.
\end{align*}

\newpage
\noindent
In \autoref{fig:aufgb} sind die Messwerte des Photostroms der orangenen Spektrallinie für eine Beschleunigungsspannung von $0$ bis $\SI{19}{\volt}$ zu sehen.
\begin{figure}[H]
  \centering
  \includegraphics[width=\textwidth]{aufgb.pdf}
  \caption{Photostrom in Abhängigkeit der Bremsspannung für die orangene Spektrallinie.}
  \label{fig:aufgb}
\end{figure}
\noindent
Die einzelnen Messwerte bilden den erwarteten Kurvenverlauf. Für höher werdende Beschleunigungsspannungen erreicht der Photostrom einen Sättigungswert, der nicht überschritten wird.
Dies liegt daran, dass die Intensität des Lichtes unverändert bleibt und so die Anzahl an Elektronen, die aus dem Material gelöst werden, begrentzt ist.
\newline
Dass sich die Kurve schon bei Bremsspannungen kleiner der Gegenspannung Null nährt liegt daran, dass die Elektronen nicht monoenergetisch sind, sondern Energien auf einem
Energieitervall besitzten, dessen obere Grenze durch die Gegenspannung definiert ist. Ein Größteil der Elektronen hat somit eine Energie darunter.
