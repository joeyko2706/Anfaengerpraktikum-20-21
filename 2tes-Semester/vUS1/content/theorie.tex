\section{Zielsetzung}
\label{sec:Zielsetzung}

In diesem Versuch sollen die grundlegenden Eigenschaften der Ultraschallechographie untersucht und angewendet werden.

\section{Theorie}
\label{sec:Theorie}
Menschen können in einem Frequenzintervall von ca. $\SI{16}{\hertz}$ bis $\SI{20}{\kilo\hertz}$ hören. Der Frequenzbereich darüber bis $\SI{1}{\giga\hertz}$ wird als
Ultraschall und der jenseits von $\SI{1}{\giga\hertz}$ als Hyperschall bezeichnet. Frequenzen unterhalb des für Menschen hörbaren Bereichs werden Infraschall gennant.
\newline
Zur Erzeugung des Schalls wird sich der reziproke piezo-elektrische-Effekt zu Nutzen gemacht. Hierfür wird ein geeigneter piezo-elektrischer Kristall in einem
elektrischen Wechselfeld zu Schwingungen angeregt, wobei dieser Ultraschallwellen abstrahlt. Der Kristall kann ebenso genutzt werden, um Schallwellen zu empfangen,
da diese ihn wiederum zu Schwingungen anregen.\\
\\

\label{sec:Schallwellen}
Bei Schallwellen handelt es sich in Gasen und Flüssigkeiten um longitudinale Wellen, die sich in Form von Druckschwankungen fortbewegt.
Eine Schallwelle mit Ausbreitung in x-Richtung lässt sich durch
\begin{equation}
    p(x, t) = p_0 + v_0 Z \cos{(\omega t - kx)}
\end{equation}
beschreiben. Dabei ist $p_0$ der Normaldruck, $v_0$ die Schallschnelle, und $Z=\rho c$ die Akustische
Impedanz mit der Schallgeschwindigkeit $c$ und der Dichte $\rho$ des durchschallten Materials.
Die Schallgeschwindigkeit in Flüssigkeiten lässt sich in Abhängigkeit ihrer Kompressibilität $\kappa$ und ihrer
Dichte $\rho$ durch
\begin{equation}
    c_{\mathrm{Fl}} = \sqrt{\frac{1}{\kappa \cdot \rho}}
\end{equation}
ausdrücken. \\
In Festkörpern können Schallwellen aufgrund von Schubspannungen auch als Transversalwellen auftreten. In diesem Fall wird 
$\kappa$ durch den Zusammenhang zum Elastizitätsmodul $E$ ersetzt. Es folgt also
\begin{equation}
    c_{\mathrm{Fe}} = \sqrt{\frac{E}{\rho}}.
\end{equation}
Ein gewisser Teil der Energie des Schalls geht bei der Ausbreitung durch Absorption verloren. Für seine Intensität gilt dabei
\begin{equation}
    \label{eqn:Dämpfung}
    I(x) = I_0 \cdot e^{-\alpha x},
\end{equation}
wobei $\alpha$ der Absorptionskoeffizient der Schallamplitude und $I_0$ die Intensität des ausgehenden 
Schallimpulses ist.\\
\\
Wenn eine Schallwelle auf eine Grenzfläche trifft, wird ein Teil von ihr reflektiert. Um welchen Anteil von der ursprünglichen Welle es sich bei der
reflektierten Welle handelt, ist durch den Reflexionskoeffizienten $R$ bestimmt. Ein Ausdruck für eben diesen ist gegeben durch
\begin{equation}
    R = \left(\frac{Z_{\symup{1}}- Z_{\symup{2}}}{Z_{\symup{1}}+Z_{\symup{2}}}\right)^2.
\end{equation}
Hierbei sind $Z_{\symup{1}}$ und $Z_{\symup{2}}$ die jeweiligen akustischen Impedanzen der beiden Materialien. \\
Für den Transmissionskoeffizienten, der den transmittierten Anteil der Welle angibt, gilt somit offensichtlich
\begin{equation}
    T = 1 - R.
\end{equation}


\label{sec:Ultraschallverfahren}
In dem Versuch werden zwei Verfahren der Ultraschalltechnik angewendet.\\ \\
Beim ersten handelt es sich um das Durchschallungsverfahren. Hierbei werden sowohl ein Sender, als auch ein Empfänger verwendet. Diese werden an unterschiedliche
Seiten der Probe angelegt. Es wird anschließend ein Schallimpuls durch die Probe gesendet. Anhand des Intensitätsabfalls kann dann bestimmt werden, ob die Probe eine
Fehlstelle enthält. Dabei kann die Position dieser jedoch nicht bestimmt werden.\\ \\
Das zweite Verfahren ist das Impuls-Echo-Verfahren. Hierbei wird nur ein Sender verwendet, der ebenfalls als Empfänger fungiert. Der ausgesendete Schallimpulls wird dabei
an einer Grenzfläche reflektiet und kann so wieder vom Empfänger aufgenommen werden. Die Höhe des Echos kann dabei zur Bestimmung der Größe einer Fehlstelle verwendet
werden. Bei bekannter Geschwindigkeit des Schalls im Medium kann durch
\begin{equation}
    \label{eqn:Strecke}
    s = \frac{ct}{2}
\end{equation}
die Position der Fehlstelle über die Laufzeit $t$ des Echos bestimmt werden.
