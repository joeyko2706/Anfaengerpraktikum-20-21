\section{Durchführung}
\label{sec:Durchführung}

Es werden für diesen Versuch Ultraschallsonden mit einer Frequenz von $\qty{2}{\mega\hertz}$ verwendet. Diese sind über einen Verstärker an einen Computer angeschlossen.
Auf dem Computer werden die Messwerte mit Hilfe des Programms EchoView graphisch dargestellt und abgespeichert. Die Sonden werden vor jedem Kontakt mit Proben mit Ultraschall
Gel bestrichen. \\
\\
Zunächst wird mit einer Schieblehre die Dicke einer Acrylplatte ausgemessen. Anschließend wird ein Scan mittels des Impuls-Echo-Verfahrens durchgeführt. Aus den gemessenen
Daten kann so die Schallgeschwindigkeit bestimmt und die gemessene Dicke der Platte verifiziert werden. \\
\\
Im zweiten Teil der Messungen soll erneut die Geschwindigkeit von Schall zusammen mit der Dämpfung in Acryl bestimmt werden. Hierfür werden vier Acryl-Zylinder verschiedener
Längen verwendet. Die Längen werden zunächst erneut mit Hilfe der Schieblehre gemessen. Anschließend werden sieben Scans mit Hilfe des Impuls-Echo-Verfahrens durchgeführt. Dafür
werden auch Messungen mit mehreren Zylinder gestapelt aufgenommen. Dabei wird auch auf sämtlich Kontaktflächen zweier Zylinder jeweils Ultraschall-Gel aufgetragen. Aus den somit
gemessenen Laufzeiten kann im Anschluss die Schallgeschwindigkeit bestimmt werden. Es wird ebenfalls eine Messung mit den vier Zylindern nach dem Durchschallungsverfahren durchgeführt.
\\
Die Bestimmung der Dämpfung erfolgt durch eine erneute Messung nach dem Impuls-Echo-Verfahren. Die Messungen erfolgen für Zylinder sechs verschiedener Längen. Hierfür werden
nun jeweils die Amplituden des ausgehenden und reflektierten Pulses gemessen. Die Dämpfung ergiebt sich dann aus deren Verhältnis.\\
\\
Im letzten Teil des Versuches soll erneut das Impuls-Echo-Verfahren angewendet werden, um ein Modell eines menschlichen Auges zu untersuchen. Hierzu werden Ultraschallsonde und
Modell mit reichlich Ultraschall-Gel bedeckt. Die Sonde wird dann in unterschiedlichen Einfallswinkeln an das Auge gehalten, bis ein geeignetes Echo von der Rückwand der Retina
zu sehen ist. Aus den gemessenen Peaks können anschließend die Abmaße des Modells bestimmt werden.


