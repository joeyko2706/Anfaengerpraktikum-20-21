\section{Durchführung}
\label{sec:Durchführung}

Als erstes sollen mit einem A-Scan die Störstellen in einem Acrylblock bestimmt werden. Dazu wird der Acrylblock zunächst mit einer Schieblehre vermessen. Danach wird
destilliertes Wasser als Kontaktmittel zwischen Ultraschallsonde und Acryl aufgetragen und der Acrylblock kann vermessen werden.  Es wird eine Ultraschallsonde mit einer Senderfrequenz
von $\SI{2}{\mega\hertz}$ verwendet. Nachdem der Block von einer Seite vermessen wurde, soll er auch von der gegenüberliegenden Seite noch einmal gemessen werden. Somit können die Durchmesser
und die Positionen der Störstellen bestimmt werden.
Als nächstes soll der Block noch einmal mit einem B-Scan abgemessen werden. \newline
Als nächstes soll an einem Modell einer Brust die Lage und Größe verschiedener Tumore bestimmt werden. Dazu wird als erstes die Lage der zwei Tumore ertastet und die Gebiete
mit einem A-Scan näher untersucht, um die Geräteparameter genauer einzustellen. Hier wird als Kontaktmittel nun Ultraschallgel verwendet. Als letztes werden mit einem B-Scan entlang einer gedachten Linie die Tumore
aufgenommen.