\section{Zielsetzung}
\label{sec:Zielsetzung}

Ziel dieses Versuches ist es, das Verhalten von Rohrströhmungen mit Hilfe von Ultraschall zu untersuchen.

\section{Theorie}
\label{sec:Theorie}

Menschen können in einem Frequenzintervall von ca. $\SI{16}{\hertz}$ bis $\SI{20}{\kilo\hertz}$ hören. Der Frequenzbereich darüber bis $\SI{1}{\giga\hertz}$ wird als
Ultraschall und der jenseits von $\SI{1}{\giga\hertz}$ als Hyperschall bezeichnet. Frequenzen unterhalb des für Menschen hörbaren Bereichs werden Infraschall gennant.
\newline
Zur Erzeugung des Schalls wird sich der reziproke piezo-elektrische-Effekt zu Nutzen gemacht. Hierfür wird ein geeigneter piezo-elektrischer Kristall in einem
elektrischen Wechselfeld zu Schwingungen angeregt, wobei dieser Ultraschallwellen abstrahlt. Der Kristall kann ebenso genutzt werden, um Schallwellen zu empfangen,
da diese ihn wiederum zu Schwingungen anregen.
\newline
In diesem Versuch wird der Doppler-Effekt ausgenutzt, um die Geschwindigkeit von Rohrströhmungen zu messen. Durch die Bewegungsgeschwindigkeit der Flüssigkeit wird
die Frequenz der Schallwellen verändert. Aus der Differenz der Frequenzen lässt sich dann wiederum auf die Geschwindigkeit der Strömung schließen.
Für die Differenz der Frequenzen gilt
\begin{equation}
    \upDelta \nu = 2\nu_0 \frac{v}{c} \cos(\alpha),
    \label{eqn:df}
\end{equation}
\noindent
wobei $\nu_0$ die ursprüngliche Frequenz, $v$ die Geschwindigkeit der Strömung, $c$ die Schallgeschwindigkeit in der Flüssigkeit und $\alpha$ der Winkel unter dem die
Schallwellen auf die Röhre treffen ist.
\newline\newline
Um die verwendete Ultraschallsonde besser an die Röhre koppeln zu können und dies vor allem auch unter einfach reproduzierbaren Winkeln, wird ein Doppler-Prisma aus
Acryl verwendet. Auf Grund von Brechung innerhalb des Prismas entsprechen die Einstellwinkel des Prismas nicht den zu verwendenden Doppler-Winkeln. Aus dem
Brechungsgesetz folgt für die zu verwendenden Winkel $\alpha$
\begin{equation}
    \alpha = 90^\circ - \arcsin(\sin(\varphi)\frac{c_\text{L}}{c_\text{P}}),
    \label{eqn:alpha}
\end{equation}
\noindent
wobei $c_\text{L}$ die Schallgeschwindigkeit in der Flüssigkeit, $c_\text{P}$ die Schallgeschwindigkeit im Prismenmaterial und $\varphi$ der Einstellwinkel des Prismas ist \cite{AnleitungUS3}.