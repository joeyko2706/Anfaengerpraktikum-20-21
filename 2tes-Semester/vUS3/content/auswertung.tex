\section{Auswertung}
\label{sec:Auswertung}

Als erstes wird zu den jeweiligen Winkeln $\varphi$, die am Prisma eingestellt wurden, der zugehörige Dopplerwinkel $\vartheta$ nach Formel [REFERENZ]
berechnet und in \autoref{tab:winkel} eingetragen.
\begin{table}[H]
  \centering
  \caption{Prisma- und Dopplewinkel.}
  \label{tab:winkel}
  \begin{tabular}{c c}
    \toprule
    Prismawinkel $\varphi$ & Dopplerwinkel $\vartheta$ \\
    \midrule
    $\SI{15}{\degree}$ & $\SI{90,58}{\degree}$ \\
    $\SI{30}{\degree}$ & $\SI{88,85}{\degree}$ \\
    $\SI{45}{\degree}$ & $\SI{91,41}{\degree}$ \\
    \bottomrule
  \end{tabular}
\end{table}

\noindent
Es wird die Strömungsgeschwindigkeit in Abhängigkeit des Dopplerwinkels $\vartheta$ untersucht. Hierzu wird ein Rohr mit einem Außendurmesser von $\SI{20}{\milli\meter}$
und einem Innendurchmesser von $\SI{16}{\milli\meter}$ verwendet und an drei verschiedenen Prismawinkeln ausgewertet. Die Messwerte sind aus \autoref{tab:AufgabeA} zu entnehmen.
An der Zentrifugalpumpe, die die Strömungsgeschwindigkeit regelt, wurde in einem Bereich von $3000 \,\text{rpm}$ bis $7000 \,\text{rpm}$ in Schrittweiten von $1000 \,\text{rpm}$ gemessen. Dies entsricht den in der
Tabelle verzeichneten Abweichungen von der maximalen Leistung ($8400\, \text{rpm}$). Sie sind gegenüber der Differenz der maximalen und minimalen Frequenz $\upDelta \nu$ dargestellt.
\begin{table}[H]
  \centering
  \caption{Prisma- und Dopplewinkel.}
  \label{tab:AufgabeA}
  \begin{tabular}{c c c c c c c}
    \toprule
    Prismawinkel $\varphi$ & $35,7 \%$ & $47,62 \%$ & $59,52 \%$ & $71,43 \%$ & $83,3 \%$\\
    \midrule
    $\SI{15}{\degree}$ & $\SI{45}{\hertz}$ & $\SI{62}{\hertz}$ & $\SI{91}{\hertz}$ & $\SI{175}{\hertz}$ & $\SI{186}{\hertz}$ \\
    $\SI{30}{\degree}$ & $\SI{55}{\hertz}$ & $\SI{102}{\hertz}$ & $\SI{152}{\hertz}$ & $\SI{246}{\hertz}$ & $\SI{355}{\hertz}$ \\
    $\SI{45}{\degree}$ & $\SI{96}{\hertz}$ & $\SI{184}{\hertz}$ & $\SI{298}{\hertz}$ & $\SI{461}{\hertz}$ & $\SI{643}{\hertz}$ \\
    \bottomrule
  \end{tabular}
\end{table}

\noindent
Die Strömungsgeschwindigkeiten $v$ folgen nun, indem die Formel [REFERENZ] umgestellt wird. Die berechneten Strömungsgeschwindigkeiten sind in \autoref{tab:stroemungsgeschw} eingetragen.
\begin{table}[H]
  \centering
  \caption{Prisma- und Dopplewinkel.}
  \label{tab:stroemungsgeschw}
  \begin{tabular}{c c c c c c c}
    \toprule
    Prismawinkel $\varphi$ & $35,7 \%$ & $47,62 \%$ & $59,52 \%$ & $71,43 \%$ & $83,3 \%$\\
    \midrule
    $\SI{15}{\degree}$ & $\SI{855}{\meter\per\second}$ & $\SI{62}{\meter\per\second}$ & $\SI{91}{\meter\per\second}$ & $\SI{175}{\meter\per\second}$ & $\SI{186}{\meter\per\second}$ \\
    $\SI{30}{\degree}$ & $\SI{55}{\meter\per\second}$ & $\SI{102}{\meter\per\second}$ & $\SI{152}{\meter\per\second}$ & $\SI{246}{\meter\per\second}$ & $\SI{355}{\meter\per\second}$ \\
    $\SI{45}{\degree}$ & $\SI{96}{\meter\per\second}$ & $\SI{184}{\meter\per\second}$ & $\SI{298}{\meter\per\second}$ & $\SI{461}{\meter\per\second}$ & $\SI{643}{\meter\per\second}$ \\
    \bottomrule
  \end{tabular}
\end{table}


\begin{table}[H]
  \centering
  \caption{Messwerte des zweiten Aufgabenteils bei 70 \% Leistung (6.000 rpm) des Gerätes.}
  \label{tab:Werte1}
  \begin{tabular}{c c c}
    \toprule
    Tiefe / $\si{\micro\second}$ & Signalstärke / $\SI{1000}{\square\volt\per\second}$ & Fließgeschwindigkeit / $\si{\centi\meter\per\second}$ \\
    \midrule
    12,0 & 5 & 181,5 \\
    12,5 & 6 & 124,2 \\
    13,0 & 7 & 39,8 \\
    13,5 & 8 & 44,6 \\
    14,0 & 13 & 50,9 \\
    14,5 & 16 & 54,1 \\
    15,0 & 20  & 60,5 \\
    15,5 & 10 & 73,2 \\
    16,0 & 7 & 66,9 \\
    16,5 & 10 & 66,9 \\
    17,0 & 11 & 57,3 \\
    17,5 & 6 & 57,3 \\
    18,0 & 9 & 44,6 \\
    18,5 & 7 & 50,9 \\
    19,0 & 7 & 54,1 \\
    19,5 & 7 & 54,1 \\
    20,0 & 6 & 60,5 \\
    \bottomrule
  \end{tabular}
\end{table}

\begin{table}[H]
  \centering
  \caption{Messwerte des zweiten Aufgabenteils bei 45 \% Leistung (3.870 rpm) des Gerätes.}
  \label{tab:Werte2}
  \begin{tabular}{c c c}
    \toprule
    Tiefe / $\si{\micro\second}$ & Signalstärke / $\SI{1000}{\square\volt\per\second}$ & Fließgeschwindigkeit / $\si{\centi\meter\per\second}$ \\
    \midrule
    12,0 & 4 & 342,2 \\
    12,5 & 5 & 79,6 \\
    13,0 & 6 & 38,2 \\
    13,5 & 7 & 28,7 \\
    14,0 & 8 & 28,7 \\
    14,5 & 9 & 28,7 \\
    15,0 & 14 & 30,2 \\
    15,5 & 17 & 30,2 \\
    16,0 & 16 & 31,8 \\
    16,5 & 12 & 28,7 \\
    17,0 & 16 & 27,1 \\
    17,5 & 11 & 25,5 \\
    18,0 & 8 & 25,5 \\
    18,5 & 7 & 28,7 \\
    19,0 & 8 & 28,7 \\
    19,5 & 8 & 31,8 \\
    20,0 & 6 & 36,6 \\
    \bottomrule
  \end{tabular}
\end{table}