\section{Auswertung}
\label{sec:Auswertung}

\begin{figure}[H]
    \centering
    \includegraphics[width=\textwidth]{build/plot1.pdf}
    \caption{Grafik.}
    \label{fig:plot1}
  \end{figure}

\begin{table}[H]
    \centering
    \caption{Messdaten mit dem sich daraus ergebenden Brechungsindex bei parallel polarisiertem Licht.}
    \label{tab:ppolMess}
    \begin{tabular}{c c c}
        \toprule
        Winkel in Grad & Photostrom / $\si{\micro\ampere}$  & Brechungsindex $n$ \\
        \midrule
        10.0  &  4.7  &  3.903  \\
        12.0  &  4.4  &  -4.226  \\
        14.0  &  3.8  &  -28.04  \\
        16.0  &  3.1  &  4.264  \\
        18.0  &  2.9  &  -6.544  \\
        20.0  &  2.2  &  -11.134  \\
        22.0  &  2.1  &  4.759  \\
        24.0  &  1.6  &  -11.698  \\
        26.0  &  1.2  &  -7.973  \\
        28.0  &  0.9  &  5.555  \\
        30.0  &  0.8  &  -35.852  \\
        32.0  &  0.9  &  -6.843  \\
        34.0  &  0.8  &  6.93  \\
        36.0  &  0.8  &  47.267  \\
        38.0  &  0.75  &  -6.503  \\
        40.0  &  0.7  &  9.551  \\
        44.0  &  0.5  &  -6.677  \\
        48.0  &  0.44  &  10.886  \\
        50.0  &  0.42  &  -7.369  \\
        52.0  &  0.45  &  44.481  \\
        54.0  &  0.41  &  8.907  \\
        56.0  &  0.39  &  -8.814  \\
        58.0  &  0.37  &  -64.206  \\
        60.0  &  0.35  &  8.17  \\
        62.0  &  0.35  &  -11.743  \\
        64.0  &  0.36  &  -20.504  \\
        66.0  &  0.35  &  8.161  \\
        68.0  &  0.33  &  -18.811  \\
        70.0  &  0.3  &  -13.262  \\
        72.0  &  0.3  &  8.806  \\
        74.0  &  0.28  &  -50.281  \\
        76.0  &  0.25  &  -10.613  \\
        78.0  &  0.22  &  10.332  \\
        80.0  &  0.22  &  81.3  \\
        82.0  &  0.25  &  -9.567  \\
        84.0  &  0.42  &  13.522  \\
        86.0  &  0.83  &  24.251  \\
        88.0  &  0.52  &  -9.417  \\
        \bottomrule
    \end{tabular}
\end{table}

\noindent
Aus diesen Werten folgt der Brechungsindex zu
\begin{align*}
    1,2904 \pm 25,0
\end{align*}