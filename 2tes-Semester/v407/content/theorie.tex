\section{Zielsetzung}
\label{sec:Zielsetzung}

Ziel des Versuches ist es, die Abhängigkeit der Intensität des, an einem Silizium-Spiegel reflektierten, Lichtes vom Eintrittswinkel und der Polarisation zu bestimmen. 

\section{Theorie}
\label{sec:Theorie}

Beim Durchqueren von Grenzflächen zweier Medien mit verschiedenen Brechungsindexen, wird das sich ausbreitende Licht in den meisten Fällen sowohl reflektiert, als
auch gebrochen. Um zu bestimmen, welchen Anteil der ursprünglichen Intensität des einfallenden Lichtes das reflektierte und das gebrochene Licht haben, wird
zunächst ein allgemeiner Ausdruck für die Strahlungsleistung von Lichtwellen benötigt.
\newline
Aus den Maxwell Gleichungen und der Elektrizitätslehre folgt der Audruck 
\begin{equation}
    \vec{S} = \vec{E} \times \vec{H},
\end{equation}
wobei $\vec{E}$ die elektrische und $\vec{H}$ die magnetische Feldstärke einer elektromagnetischen Welle beschreiben. $\vec{S}$ wird als Poynting-Vektor bezeichnet
und hat die Dimension $\frac{\text{Leistung}}{\text{Fläche}}$. Er beschreibt den Transport von Energie, da er in die Ausbreitungsrichtung der elektromagnetischen
Welle zeigt und sein Betrag dem der Energie der Strahlung entspricht.

\begin{equation}
    S_\text{e} \cos(\alpha) = S_\text{r} \cos(\alpha) + S_\text{d} \cos(\alpha)
\end{equation}

\begin{equation}
    n = \frac{c}{v}
\end{equation}

\begin{equation}
    n = \frac{\sin(\alpha)}{\sin(\beta)}
\end{equation}

\begin{equation}
    \vec{E_{\text{r}\perp}}(\alpha) = \vec{E_{\text{e}\perp}} \frac{(\sqrt{n^2 - \sin^2(\alpha)} - \cos(\alpha))^2}{n^2 - 1}
    \label{eqn:fresnelsenk}
\end{equation}

\begin{equation}
    \vec{E_{\text{r}\parallel}}(\alpha) = \vec{E_{\text{e}\parallel}} \frac{n^2 \cos(\alpha) + \sqrt{n^2} - \sin^2(\alpha)}{n^2 \cos(\alpha) - \sqrt{n^2} - \sin^2(\alpha)}
    \label{eqn:fresnelpara}
\end{equation}

\cite{sample}
