\section{Diskussion}
\label{sec:Diskussion}

In \autoref{tab:abweich} werden die gemessenen Brechungsindizes mit dem Literaturwert von $n_{\text{Lit}}=3,353$ verglichen.
\begin{table}[H]
    \centering
    \caption{Abweichungen der gemessenen Brechungsindizes zum Literaturwert.}
    \label{tab:abweich}
    \begin{tabular}{c c c}
        \toprule
         & Messwert & Abweichung \\
        \midrule
        $n_{\parallel}$ & $1,290 \pm 24,616$ & 61,52 \% \\
        $n_\perp$ & $1.043 \pm 0,031$ & 68,89 \% \\
        $n_{\text{Brewster}}$ & $4,705$ & 40,32 \% \\
        \bottomrule
    \end{tabular}
\end{table}

\noindent
Die Abweichungen der Messwerte von den Theoriewerten ist eher als hoch einzuschätzen und auch die grafische Auswertung in \autoref{fig:plot} ist sehr
ungenau. Damit lässt sich sagen, dass die Theorie nicht bestätigt werden kann. Im folgenden \autoref{subsec:fehler} werden die möglichen Fehlerquellen noch einmal diskutiert.

\subsection{Diskussion der Fehlerquellen}
\label{subsec:fehler}

Eine mögliche Fehlerquelle ist der Aufbau des Versuches, da hier der Lichtstrahl möglichst geradlinig einzustellen ist.\newline
Eine weitere mögliche Fehlerquelle ist das von außen in die Photozelle einströmende Licht. So drang Sonnenlicht in die Photozelle ein und verfälschte somit den gemessenen
Photostrom, der von dem reflektierten Lichtstrahl kam. Dieser Fehler ist als besonders hoch einzuschätzen, da der Versuch direkt am Fenster stand und somit besonders viel
Licht die Messungen verfälschen konnte. \newline
Die zweite große Fehlerquelle ist das Ausrichten des Laserstrahls auf die Spaltmitte der Photozelle. Wenn hier nicht jedes mal die richtige Position eingestellt wird, kann es zu großen
Abweichungen beim Photostrom kommen. Das liegt daran, dass dann nicht der komplette Lichtstrahl die Photokathode trifft und deshalb weniger Elektronen herausgelöst werden können.
In der Folge wird der Photostrom niedriger. Besonders hohen Unsicherheiten unterlag der Versuch dadurch, dass sich die Photozelle nicht gut mit der drehenden Platte mitdrehte.
Wurde ein neuer Winkel eingestellt, so musste die Zelle nachjustiert werden, wodurch es zu hohen Schwankungen zwischen den Messwerten kam.