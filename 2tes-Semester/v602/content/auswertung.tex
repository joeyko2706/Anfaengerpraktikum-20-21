\section{Auswertung}
\label{sec:Auswertung}

Im folgenden werden die in \autoref{sec:Durchführung} aufgeführten Schritte ausgewertet und mit den Theoriewerten verglichen.

\subsection{Auswrtung des Emissionsspektrums der Kupfer-Röntgenröhre}
In \autoref{fig:plot1} ist das aufgezeichnete Emissionsspektrum der Kupfer-Röntgenröhre aufgezeichnet. Es sind außerdem die $K_\alpha$- und die $K_\beta$-Linien
eingezeichnet. Der Bremsberg ist dabei das gesamte Spektrum, mit Ausnahme der beiden Peaks, die jeweils die Absoptionskanten darstellen. 

\begin{figure}[H]
  \centering
  \includegraphics{build/plot1.pdf}
  \caption{Grafische Darstellung des Emissionsspektrums.}
  \label{fig:plot1}
\end{figure}

Die Halbwertsbreite wird optisch aus einer grafischen Darstellung abgelesen, da das mit \autoref{fig:plot1} jedoch nur schwer zu erreichen ist, wird der Bildbereich
der beiden Absorptionskanten etwas vergrößert und in \autoref{fig:plot2} nocheinmal dargestellt. Die vertikal eingezeichneten Linien sind dabei die Halbwertsbreiten.

\begin{figure}[H]
  \centering
  \includegraphics{build/plot2.pdf}
  \caption{Grafische Darstellung der beiden Absorptionskanten.}
  \label{fig:plot2}
\end{figure}

Die abgelesenen Winkel der beiden Absoptionskanten werden nun mithilfe der Bragg-Bedingung \eqref{eqn:bragg} die minimale Wellenlänge und daraus die maximale Energie
berechnet. Die resultierenden Werte sind in \autoref{tab:werte1} eingetragen.
\begin{table}[H]
  \caption{Theoretische und experimentell bestimmte Werte des absorptionsspektrums.}
  \label{tab:werte1}
  \centering
  \begin{tabular}{c c c}
      \toprule
       & $K_\alpha$ & $K_\beta$\\
      \midrule
      Maximum & $22,5 ~ 2\vartheta$ & $20,2 ~ 2\vartheta$ \\
      theoretische Energie & $\SI{8047,823}{\eV}$ & $\SI{8905,413}{\eV}$ \\
      gemessene Energie & $\SI{8043,355}{\eV}$ & $\SI{8914,204}{\eV}$ \\
      Abweichung & $0,056 \%$ & $0,0987 \%$ \\
      Halbwertsbreite & $\SI{0,42}{\degree}$ & $\SI{0,451}{\degree}$ \\
      \bottomrule
    \end{tabular}
\end{table}

Das Auflösevermögen $A$ der Apparatur wird durch die Gleichung
\begin{align*}
  A = \frac{E_{\text K}}{\upDelta E_{\text{FWHM}}},
\end{align*}
wobei $E_{\text K}$ die Energie der Absorptionskanten und $\upDelta E_{\text{FWHM}}$ die Energie der Halbwertsbreite beschreibt.
Es ergeben sich damit die folgenden Werte,
\begin{align*}
  A_{K_\alpha} &= ,\\
  A_{K_\beta} &= .
\end{align*}