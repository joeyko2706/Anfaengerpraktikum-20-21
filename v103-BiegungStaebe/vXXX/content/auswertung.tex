\section{Auswertung}
\label{sec:Auswertung}

\subsection{Elastizitätsmodul des eckigen Stabes}
\label{subsec:elastiEckig}
Es wird das Elastizitätsmodul $E$ eines eckigen Stabes, dessen Abmaße \autoref{tab:eckigStab} zu entnehmen sind, berechnet.
\begin{table}[H]
  \centering
  \caption{Abmaße des eckigen Stabes.}
  \label{tab:eckigStab}
  \begin{tabular}{c c c}
    \toprule
    m / g & l / mm & d / mm \\
    \midrule
    535.4 & 10.0 & 602.0 \\
    536.2 & 10.0 & 602.0 \\
    536.3 & 10.0 & 602.0 \\
    535.7 & 10.0 & 602.0 \\
    536.1 & 10.0 & 602.0 \\
    \bottomrule
  \end{tabular}
\end{table}

Aus (\ref{eqn:Durchbiegung}) folgt ein Zusammenhang zwischen dem Drehmoment $D(x)$ und dem Elastizitätsmodul $E$. Um letzteres zu bestimmen
wird eine Hilfsvariable $\eta(x)$ eingeführt, mit 
\begin{align*}
  \eta(x) = Lx^2- \frac 13 x^3.
\end{align*}
Wird $\eta(x)$ in (\ref{eqn:Durchbiegung}) eingesetzt, folgt ein linearer Zusammenhang zwischen $D(x)$ und $\eta(x)$,
\begin{align}
  D(\eta) = \frac{F}{2EI}\eta.
  \label{eqn:rel1}
\end{align}
Mithilfe einer linearen Regressionskurve wird ein Wert für die Steigung $m$ ermittelt, aus dem das Elastizitätsmodul und (\ref{eqn:rel1}) 
bestimmt werden kann.

Das Gewicht des Stabes ergibt sich mit \autoref{tab:eckigStab} zu $\bar m_{\text{quadr}} = \SI{535,7 (0,7)}{\gram}$, woraus die Gewichtskraft
zu 


\begin{table}[H]
  \centering
  \caption{Messung eines quadratischen Stabes bei einseitger Einspannung ($\text G = \SI{750}{\gram}$).}
  \label{tab:werte1}
  \begin{tabular}{c c c}
    \toprule
    x / $\si{\centi\meter} $ & $ D_0(x) / \si{\centi\meter}$ & $D_{\text G}(x) / \si{\milli\meter}$ \\
    \midrule
    3 & 7,79 & 7,76 \\
    6 & 7,79 & 7,70 \\
    9 & 7,78 & 7,57 \\
    12 & 7,71 & 7,43 \\
    15 & 7,64 & 7,20 \\
    18 & 7,58 & 6,97 \\
    21 & 7,58 & 6,76 \\
    24 & 7,46 & 6,46 \\
    27 & 7,36 & 6,13 \\
    30 & 7,27 & 5,82 \\
    33 & 7,19 & 5,48 \\
    36 & 7,13 & 5,12 \\
    39 & 6,95 & 4,71 \\
    42 & 6,80 & 4,31 \\
    45 & 6,63 & 3,85 \\
    48 & 6,49 & 3,49 \\
    \bottomrule
  \end{tabular}
\end{table}

\begin{table}[H]
  \centering
  \caption{Messung eines zylindrischen Stabes bei einseitger Einspannung ($\text G = \SI{750}{\gram}$).}
  \label{tab:werte2}
  \begin{tabular}{c c c}
    \toprule
    x / $\si{\centi\meter} $ & $ D_0(x) / \si{\centi\meter}$ & $D_{\text G}(x) / \si{\milli\meter}$ \\
    \midrule
    3 & 8,13 & 8,07 \\
    6 & 8,21 & 8,00 \\
    9 & 8,27 & 7,98 \\
    12 & 8,34 & 7,87 \\
    15 & 8,41 & 7,72 \\
    18 & 8,45 & 7,44 \\
    21 & 8,57 & 7,32 \\
    24 & 8,63 & 7,11 \\
    27 & 8,74 & 6,84 \\
    30 & 8,81 & 6,60 \\
    33 & 8,92 & 6,32 \\
    36 & 9,01 & 5,99 \\
    39 & 9,09 & 5,65 \\
    42 & 9,10 & 5,37 \\
    45 & 9,10 & 4,93 \\
    48 & 9,14 & 4,64 \\
    \bottomrule
  \end{tabular}
\end{table}

\begin{table}[H]
  \centering
  \caption{Messung eines quadratischen Stabes bei beidseitiger Einspannung ($\text G = \SI{1550}{\gram}$).}
  \label{tab:werte3}
  \begin{tabular}{c c c}
    \toprule
    x / $\si{\centi\meter} $ & $ D_0(x) / \si{\centi\meter}$ & $D_{\text G}(x) / \si{\milli\meter}$ \\
    \midrule
    3 & 8,90 & 8,75 \\
    6 & 8,99 & 8,76 \\
    9 & 9,07 & 8,76 \\
    12 & 9,14 & 8,76 \\
    15 & 9,22 & 8,77 \\
    18 & 9,31 & 8,74 \\
    21 & 9,39 & 8,83 \\
    24 & 9,46 & 8,87 \\
    27 & 9,55 & 8,92 \\
    30 & 7,19 & 7,88 \\
    33 & 7,11 & 7,67 \\
    36 & 8,02 & 7,53 \\
    39 & 8,13 & 7,32 \\
    42 & 8,11 & 7,05 \\
    45 & 8,38 & 8,05 \\
    48 & 8,45 & 8,17 \\
    51 & 8,58 & 8,45 \\
    54 & 8,73 & 8,68 \\
    \bottomrule
  \end{tabular}
\end{table}

\begin{table}[H]
  \centering
  \caption{Messung eines zylindrischen Stabes bei beidseitiger Einspannung ($\text G = \SI{1550}{\gram}$).}
  \label{tab:werte4}
  \begin{tabular}{c c c}
    \toprule
    x / $\si{\centi\meter} $ & $ D_0(x) / \si{\centi\meter}$ & $D_{\text G}(x) / \si{\milli\meter}$ \\
    \midrule
    3 & 9,93 & 8,67 \\
    6 & 9,95 & 8,59 \\
    9 & 9,97 & 8,44 \\
    12 & 9,86 & 8,30 \\
    15 & 9,82 & 8,12 \\
    18 & 9,67 & 7,85 \\
    21 & 9,60 & 7,74 \\
    24 & 9,52 & 7,63 \\
    27 & 9,36 & 7,51 \\
    30 & 7,50 & 6,63 \\
    33 & 7,38 & 6,56 \\
    36 & 7,26 & 6,54 \\
    39 & 7,17 & 6,51 \\
    42 & 7,04 & 6,49 \\
    45 & 6,95 & 6,49 \\
    48 & 6,68 & 6,48 \\
    51 & 6,72 & 6,53 \\
    54 & 6,51 & 6,60 \\
    \bottomrule
  \end{tabular}
\end{table}