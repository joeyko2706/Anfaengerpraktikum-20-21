\newpage
\section{Diskussion}
\label{sec:Diskussion}


Die Messwerte unterscheiden sich zum Teil recht stark von den Theoriewerten, was vermutlich daran liegt, dass große Messfehler
in die Berechnung der Werte eingegangen ist. So wurde beim Messen der Durchmesser der jeweiligen Körperteile nicht auf die unterschiedlich
breiten Einzelteile eingegangen (beim Arm zum Beispiel bei Ober- und Unterarm), sondern ein Mittelwert für den Durchmesser des gesamten
Körperteiles genommen.

Ein weiterer großer Faktor für Abweichungen von den Theoriewerten ist das Messen der Periodendauer. Zum Stoppen der Zeit einer 
Periodendauer wurde eine Stoppuhr verwendet, die manuell ausgelöst und gestoppt werden musste. Durch menschliche Verfehlungen, wie zum Beispiel
der Reaktionszeit, wurden die Zeiten verfälscht. Es wurde zwar versucht durch eine höhere Anzahl an Messungen dieser Unsicherheit vorzubeugen, jedoch
spielt sie dennoch in das Endergebnis mit ein. 

Die Abweichung bei der Bestimmung des Trägheitsmomentes der Kugel beträgt $17.6953 \%$ und die bei dem Zylinder $16.2894 \%$. Es läasst sich,
unter Berücksichtigung der eingegangenen Fehler sagen, dass der Wert der Abweichung dennoch recht groß ist. Eigentlich war zu erwarten, dass sich die
prozentuale Abweichung auf unter $10 \%$ beläuft. Entweder sind, entgegen den Erwartungen, schlechte Messwerte vorgenommen worden, oder es handelt
sich um einen systematischen Fehler zum Beispiel von der Messapparatur.

Die Abweichungen bei der Bestimmung der Trägheitsmomente der Puppe in verschiedenen Stellungen ist mit $41.33 \%$ bei der ersten Stellung und mit
$37.49 \%$ allerdings noch höher. Hier muss man allerdings die Näherungen mit einbeziehen, die für die Puppe getroffen wurden. Es wurden alle
Körperteile, auch der Kopf, auf Zylinder approximiert. Auch wurde für die Gliedmaßen nicht beide einzelne Glieder gemessen und dessen Trägheitsmomente
berechnet, sondern wurde für ein Glied das Trägheitsmoment exemplarisch bestimmt und in dem Gesamtträgheitsmoment doppelt einbezogen.

Da bei der Puppe so viele Approximierungen vorgenommen wurden ist allerdings daon auszugehen, dass der experimentell bestimmte Wert näher am 
tatsächlichen Wert ist, als der theoretisch bestimmte Wert.