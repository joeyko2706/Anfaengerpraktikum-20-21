\section{Daten}
\label{sec:Daten}

\subsection{Tabellen}


\begin{table}
    \centering
    \caption{Winkelrichtgröße $D$}
    \label{tab:winkelrichtgroesse}
    \begin{tabular}{c c}
        \toprule
        Auslenkwinkel $ \varphi \;/\; \text{DEG}\si{\degree}$ & $F \;/\; \si{\newton}$ \\
        \midrule
        70 & 0.20 \\
        80 & 0,24 \\
        90 & 0,28 \\
        100 & 0,33 \\
        110 & 0,35 \\
        120 & 0,39 \\
        130 & 0,41 \\
        140 & 0,47 \\
        150 & 0,49 \\
        160 & 0,52 \\
        170 & 0,55 \\
        \bottomrule
    \end{tabular}
\end{table}


\begin{table}
   \centering
    \caption{Eigenträgheitsmoment $I_D$}
    \label{tab:eigentraegheitsmoment}
    \begin{tabular}{c c}
        \toprule
        Schwingungsdauer $ T \;/\; \si{\second}$ & Abstand $a \;/\; \si{\milli\meter}$ \\
        \midrule
        3,08 & 60 \\
        3,40 & 80 \\
        4,32 & 100 \\
        4,83 & 120 \\
        4,32 & 140 \\
        4,83 & 160 \\
        5,37 & 180 \\
        5,76 & 200 \\
        6,28 & 220 \\
        7,36 & 240 \\
        \bottomrule
    \end{tabular}
\end{table}

\begin{table}
    \centering
     \caption{Trägheitsmomente der Körper}
     \label{tab:traegheitsmomente}
     \begin{tabular}{c c}
        \toprule
        $ T_{\text{Zylinder}} \;/\; \si{\second}$ & $ T_{\text{Kugel}} \;/\; \si{\second}$ \\
        \midrule
        0,876 & 1,706 \\
        0,856 & 1,724 \\
        0,828 & 1,708 \\
        0,848 & 1,692 \\
        0,843 & 1,691 \\
        0,847 & 1,699 \\
        0,804 & 1,684 \\
        0,822 & 1,677 \\
        0,862 & 1,697 \\
        0,828 & 1,687 \\
        \bottomrule
     \end{tabular}
 \end{table}

\begin{table}
\centering
    \caption{Durchmesser Körperteile}
    \label{tab:durchmesser}
    \begin{tabular}{c c c c}
    \toprule
    $D_\text{Arm} \;/\; \si{\centi\meter}$ & $D_\text{Kopf} \;/\; \si{\centi\meter}$ & $D_\text{Bein} \;/\; \si{\centi\meter}$ & $D_\text{Torso} \;/\; \si{\centi\meter}$ \\
    \midrule
    13,3 & 17,6 & 13,4 & 40,0 \\
    16,1 & 19,0 & 16,5 & 33,4 \\
    14,0 & 21,7 & 17,2 & 28,1 \\
    16,8 & 30,6 & 16,0 & 36,4 \\
    11,2 & 32,2 & 20,9 & 36,5 \\
    \bottomrule
    \end{tabular}
\end{table}

$T_1$ entspricht der Stellung 1 unter Auslenkum um $90°$.
$T_2$ entspricht der Stellung 1 unter Auslenkum um $120°$.
$T_3$ entspricht der Stellung 2 unter Auslenkum um $90°$.
$T_4$ entspricht der Stellung 2 unter Auslenkum um $120°$.

\begin{table}
    \centering
        \caption{Schwingungsdauer}
        \label{tab:schwingdauer}
        \begin{tabular}{c c c c}
        \toprule
        $T_1 \;/\; \si{\second}$ & $T_2 \;/\; \si{\second}$ & $T_3 \;/\; \si{\second}$ & $T_4 \;/\; \si{\second}$ \\
        \midrule
        0,792 & 0,790 & 1,094 & 1,070 \\
        0,816 & 0,906 & 1,148 & 1,098 \\
        0,840 & 0,784 & 1,096 & 1,096 \\
        0,798 & 0,784 & 1,058 & 1,084 \\
        0,816 & 0,812 & 1,144 & 1,198 \\
        \bottomrule
    \end{tabular}
\end{table}