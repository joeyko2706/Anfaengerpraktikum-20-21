\section{Auswertung}
\label{sec:Auswertung}

Im folgenden werden die Mittelwerte mit
\begin{equation}
  \label{eqn:mittelwert}
  \bar x = \frac 1n \sum_i^n x_i
\end{equation}
und die Standardabweichung mit
\begin{equation}
  \label{eqn:stdabweichung}
  \Delta\bar x = \frac{1}{n(n-1)}\sum_i^n (x_i- \bar x)^2
\end{equation}
berechnet.
Aus Werten mit Unsicherheiten lassen sich mithilfe der Gauß'schen Fehlerfortpflanzung, die daraus abgeleiteten Größen berechnen, die
folgendermaßen definiert ist:
\begin{equation}
  \label{eqn:gauß}
  \Delta f= \sqrt{\sum_i^n \Big(\frac{\partial f}{\partial x_i}\cdot x_i \Big)^2}
\end{equation}


\subsection{Apparatekonstante}
\label{sec:Apparatekonstante}
Die Berechnung der Winkelrichtgröße $D$ berechnet sich über (\autoref{eqn:winkelrichtgr}) mit dem konstanten Abstand 
$a= 10\si{\centi\meter}$ zu den folgenden Werten.

\begin{table}
  \centering
  \caption{Winkelrichtgröße $D$}
  \label{tab:winkelrichtgroesse}
  \begin{tabular}{c c c}
      \toprule
      Auslenkung $ \varphi \;/\; \text{DEG}$ & $F \;/\; \si{\newton}$ & $D \;/\; \si{\milli\newton\meter}$\\
      \midrule
      70 & 0.20 & 0.286 \\
      80 & 0,24 & 0.300 \\
      90 & 0,28 & 0.311 \\
      100 & 0,33 & 0.330 \\
      110 & 0,35 & 0.318 \\
      120 & 0,39 & 0.325 \\ 
      130 & 0,41 & 0.315 \\
      140 & 0,47 & 0.336 \\
      150 & 0,49 & 0.327 \\
      160 & 0,52 & 0.325 \\
      170 & 0,55 & 0.324 \\
      \bottomrule
  \end{tabular}
\end{table}

Woraus sich der Mittelwert für die Winkelrichtgröße ergibt.

\begin{equation}
  \bar D = (0.288\pm) \si{\milli\newton\meter}
\end{equation}

Das Trägheitsmoment der Drillachse ergibt sich mit zwei Gewichten.
Sie haben ein zylinderförmiges Gewicht von $m_1 = (223.2 \pm 0.1) \text{g}$ und $m_2 = (222.8 \pm 0.1) \text{g}$ und jeweils eine
Höhe $h=30\si{\milli\meter}$ und einen Durchmesser $d=35\si{\milli\meter}$.

\begin{table}
  \centering
   \caption{Messwerte zum Eigenträgheitsmoment $I_D$}
   \label{tab:eigentraegheitmess}
   \begin{tabular}{c c}
       \toprule
       Schwingungsdauer $ T \;/\; \si{\second}$ & Abstand $a \;/\; \si{\milli\meter}$ \\
       \midrule
       3,08 & 60 \\
       3,40 & 80 \\
       4,32 & 100 \\
       4,83 & 120 \\
       4,32 & 140 \\
       4,83 & 160 \\
       5,37 & 180 \\
       5,76 & 200 \\
       6,28 & 220 \\
       7,36 & 240 \\
       \bottomrule
   \end{tabular}
\end{table}


%Die Trägheitsmomente der beiden Gewichte lassen sich mithilfe (\autoref{eqn:steiner}) vereinfachen zu 
%\begin{equation}
%  I_1 = I_{\text{Zylinder}} + m_1a^2 = 1\si{\kilogram\meter^2}
%\end{equation}
%\begin{equation}
%  I_2 = I_{\text{Zylinder}} + m_2a^2 = 1\si{\kilogram\meter^2}
%\end{equation}


%Trägheitsmomente der Drillachse wird mit linearer Regression berechnet.

Die Trägheitsmomente der beiden Gewichte lassen sich mithilfe (\autoref{eqn:steiner}) vereinfachen. Zur Berechnung
des Trägheitsmomentes der Drillachse $I_D$, wird eine lineare Regression verwendet.

\begin{equation}
  \label{eqn:linReg}
  T^2 = ba^2+c
\end{equation}

Die Messwerte und die Regression sind in Abbildung (HIER ABBILDUNG EINFÜGEN) aufgetragen. Die Trägheitsmomente werden mit
\begin{center}
  $I_{\text{Stab}} = \frac{1}{12} ml^2 $ \\
  $I_{\text{Zylinder}} = m (\frac{r^2}{4} +\frac{h^2}{12}) $ \\
\end{center}
berechnet (\cite{Anleitung}). 

\subsection{Trägheitsmomente einfacher Körper}
\label{sec:Trägheitsmomente einfacher Körper}

Untersucht werden ein Zylinder und eine Kugel, dessen Rotationsachse je die Schwerpunktsachse der beiden Körper ist.
Die Trägheitsmomente des Zylinders berechnet sich, analog zu oben, durch

\begin{equation}
  I_{\text{Zylinder}} = \frac{1}{2} mR^2
\end{equation}

und das der Kugel durch

\begin{equation}
  I_{\text{Kugel}} = \frac{2}{5} mR^2 \, .
\end{equation}
Dabei ist $R$ der Radius und $m$ die Gesamtmasse der jeweiligen Körper.

Die Werte der beiden Körper sind:
\begin{center}
  $m_{\text{Kugel}} = (810.9 \pm 0.1)\si{\gram}$ \\
  $d_{\text{Kugel}} = 12.7\si{\centi\meter}$ \\
  $m_{\text{Zylinder}} = (367.8\pm 0.1)\si{\gram}$ \\
  $h = 9.00 \si{\centi\meter}$ \\
  $d_{\text{Zylinder}} = 8.72 \si{\centi\meter}$ \\
\end{center}

Damit ergeben sich die Theoriewerte der Trägheitsmomente für den Zylinder und für die Kugel.

\begin{center}
  $I_{\text{Kugel,Theorie}} = 1.306 *10^(-3) \si{\kilogram\meter^2}$ \\
  $I_{\text{Zylinder,Theorie}} = 0.349 *10^(-3) \si{\kilogram\meter^2}$ \\
\end{center}

Die gemessenen Schwingungsdauern wurden in $...$ aufgeführt. Die gemittelten Periodendauern wurden dazu benutzt
durch die Gleichung (\autoref{eqn:periode}) die Trägheitsmomente der Körper zu berechnen.

\begin{table}
  \centering
   \caption{gemessene Periodendauer}
   \label{tab:traegheitsmomente}
   \begin{tabular}{c c}
      \toprule
      $ T_{\text{Zylinder}} \;/\; \si{\second}$ & $ T_{\text{Kugel}} \;/\; \si{\second}$ \\
      \midrule
      0.876 & 1.706 \\
      0.856 & 1.724 \\
      0.828 & 1.708 \\
      0.848 & 1.692 \\
      0.843 & 1.691 \\
      0.847 & 1.699 \\
      0.804 & 1.684 \\
      0.822 & 1.677 \\
      0.862 & 1.697 \\
      0.828 & 1.687 \\
      \bottomrule
   \end{tabular}
\end{table}

Damit ergeben sich die Mittelwerte zu HIER MITTELWERTE EINFÜGEN.

Und damit die Trägheitsmomente zu HIER WERTE EINFÜGEN.

Die Abweichung der Messwerte von den Theoriewerten entspricht HIER WERTE EINFÜGEN.

\subsection{Trägheitsmoment einer Modellpuppe}
\label{sec:Trägheitsmoment einer Modellpuppe}

Zur Bestimmung des Trägheitsmomentes der Modellpuppe wird die Puppe in einzelne Teile unterteilt. Dabei wird angenommen,
dass sich die einzelnen Gliedmaßen durch Zylinder approximieren lassen können und eine homogene Massenverteilung besitzt. Die
Volumina der Einzelteile werden durch

\begin{equation}
  V = \pi r^2h
\end{equation}
bestimmt. Danach werden die Massenanteile der Einzelteile bestimmt. Die Gesamtmasse der Puppe beträgt 

\begin{center}
  $m_{ges} = (166.8\pm 0.1)\si{\gram}$.
\end{center}

Die Abmaße der Körperteile werden jeweils, über die gesamte Länge verteilt, fünf mal gemessen. Somit wird auf die unterschiedlichen
Radien der Glieder Rücksicht genommen (zum Beispiel ist beim Arm der Oberarm dicker, als der Unterarm). Die Mittelwerte werden dazu genutzt die
Trägheitsmomente auszurechnen.

\begin{table}
  \centering
    \caption{Durchmesser der Körperteile}
    \label{tab:durchmesser}
    \begin{tabular}{c c c c}
    \toprule
    $D_\text{Arm} \;/\; \si{\centi\meter}$ & $D_\text{Kopf} \;/\; \si{\centi\meter}$ & $D_\text{Bein} \;/\; \si{\centi\meter}$ & $D_\text{Torso} \;/\; \si{\centi\meter}$ \\
    \midrule
    13.3 & 17.6 & 13.4 & 40.0 \\
    16.1 & 19.0 & 16.5 & 33.4 \\
    14.0 & 21.7 & 17.2 & 28.1 \\
    16.8 & 30.6 & 16.0 & 36.4 \\
    11.2 & 32.2 & 20.9 & 36.5 \\
    \bottomrule
  \end{tabular}
\end{table}

Die Längen der Körperteile sind folgendermaßen:
\begin{center}
  $l_{\text{Arm}} = 0.0870 \si{\meter}$ \\
  $l_{\text{Unterarm}} = 0.0410 \si{\meter}$ \\
  $l_{\text{Bein}} = 0.1244 \si{\meter}$ \\
  $l_{\text{Torso}} = 0.06576 \si{\meter}$ \\
  $l_{\text{Unterarm}} = 0.01644 \si{\meter}$ \\
\end{center}


Es werden die Trägheitsmomente für zwei unterschiedliche Stellungen der Puppe berechnet. Abbilder der Stellungen der Puppe sind im folgenden
dargestellt unter \autoref{fig:stellung1} und unter \autoref{fig:stellung2}. 
Als erster wurde das Trägheitsmoment in der Stellung 1 und danach in der Stellung2 gemessen.

%--------------

%HIER WAREN URSPRÜNGLICH DIE BEIDEN BILDER DER PUPPE

%--------------

Die Abstände der Körperteile zur Drehachse ergeben sich über die Radien der Körperteile.

Damit ergeben sich die Trägheitsmomente folgendermaßen.
\begin{center}
  HIER TRÄGHEITSMOMENTE EINTRAGEN
\end{center}

Die Trägheitsmomente wurden analog zu denen der Kugel und des Zylinders berechnet. Dabei sind die Schwingungsdauern im folgenden
aufgelistet:

$T_1$ entspricht der Stellung 1 unter Auslenkung um $90°$, $T_2$ der Stellung 1 unter Auslenkung um $120°$.
Analog dazu sind $T_3$ und $T_4$ für die Stellung 2. Gemäß Gleichung (\autoref{eqn:periode}) folgt für die Puppe

%$T_3$ entspricht der Stellung 2 unter Auslenkum um $90°$.
%$T_4$ entspricht der Stellung 2 unter Auslenkum um $120°$.

\begin{equation}
  I_{\text{Puppe}} = \frac{T^2D}{4\pi} - I_D
\end{equation}

\begin{table}
    \centering
        \caption{Schwingungsdauer}
        \label{tab:schwingdauer}
        \begin{tabular}{c c c c}
        \toprule
        $T_1 \;/\; \si{\second}$ & $T_2 \;/\; \si{\second}$ & $T_3 \;/\; \si{\second}$ & $T_4 \;/\; \si{\second}$ \\
        \midrule
        0.792 & 0.790 & 1.094 & 1.070 \\
        0.816 & 0.906 & 1.148 & 1.098 \\
        0.840 & 0.784 & 1.096 & 1.096 \\
        0.798 & 0.784 & 1.058 & 1.084 \\
        0.816 & 0.812 & 1.144 & 1.198 \\
        \bottomrule
    \end{tabular}
\end{table}

Als Mittelwert ergeben sich
\begin{center}
  HIER MITTELWERTE FÜR SCHWINGUNGSDAUERN EINTRAGEN
\end{center}

woraus die Trägheitsmomente folgen %aus den Messwerten die
\begin{center}
  HIER TRÄGHEITSMOMENTE EINTRAGEN
\end{center}

Das entspricht einer Abwertung des experimentellen Wertes von dem Theoriewert um:
\begin{center}
  HIER ABWEICHUNGEN EINTRAGEN
\end{center}

%Wahrscheinlich müssen die Bilder in den Anhang, weil der die Bilder nur da einfügt, wo es passt und nicht da, wo es soll.