\section{Durchführung}
\label{sec:Durchführung}

\subsection{vorbereitende Maßnahmen}
Bevor der Versuch durchgeführt werden kann, müssen zunächst die Resonanzfrequenzen der Maschen, beziehungsweise der beiden Schwingkreise,
herausgefunden werden.
Bei der linken Maschen wird die Resonanzfrequenz mit fester Kapazität und variabler Spannungsfrequenz %umbedingt besseren Ausdruck finden
gemessen. Die Resonanzfrequenz der rechten Masche wird mit einer festen Spannungsfrequenz und variabler Kapazität  gemessen.
Die Schaltung wird gemäß EINER ABBILDUNG  aufgebaut. 

Mithilfe eines Generators wird eine Rechteckspannung angelegt, die  für die Kreisfrequenz $\omega$ im linken Schwingkreis sorgt. 
Indem Lissajous-Figuren erzeugt, die man über den XY-Betrieb am Oszilloskop erstellt, kann die Phasenverschiebung von dem Generator und 
dem Schwingkreis auf null gesetzt werden. Ist die Lissajous-Figur eine Gerade, dann sind die beiden eingehenden Signale, hier der
Generator und der Schwingkreis, in Phase. Ein Beispiel dafür ist die Abbildung \ref{fig:Lissajous}.

Es wird die Resonanzfrequenz der linken Masche festgehalten und die Resonanzfrequenz der rechten Masche ausgemessen.
Dabei wird die verstellbare Kapazität an dem Schaltkasten so eingestellt, dass die Lissajous-Figur Phasengleichheit angibt.

\subsection{Austausch der Schwingungsenergie}
Zunächst wird der Schaltplan gemäß EINER ABBILDUNG aufgebaut. Nun werde in Abhängigkeit der verstellbaren Kapazität auf dem Schaltkasten
die Kopplungskapazität $C_K$ die Schwingungsmaxima einer Schwebung gezählt. Es wird außerdem die Zeit einer Schwebungsperiode gemessen.
Der Vorgang wird für alle möglichen Kopplungskapazitäten auf dem Schaltkasten wiederholt.

\subsection{Fundamentalschwingung}
% Schaltplan umgestell gedöns
Das Oszilloskop wird wieder auf die XY-Funktion umgestellt, damit man mit Lissajous-Figuren die Frequenzen der Fundamentalschwingungen messen kann. 
Es wird eine Sinusfrequenz am Generator eingestellt.Im Anschluss werden die Frequenzen in Abhängigkeit der Kopplungskapazität gemessen, bei denen 
die Lissajous-Figuren auf eine gerade abgebildet werden. Der Vorgang wird, analog zum Aufgabenteil zuvor, für alle möglichen Kopplungskapazitäten 
auf dem Schaltkasten wiederholt.

\subsection{Strömungsverlauf}
Der Schaltplan bleibt, wie bei der Messung der Fundamentalschwingungen. Am Stromgenerator wird mithilfe des Wobbelgenerators ein Frquenzzähler
eingeschaltet. Dieser geht die Freqenzen von 20.00 kHz bis 50.00 kHz innerhalb von 0.02 Sekunden durch und bildet die Strömungsmaxima somit in
Abhängigkeit der Frequenzen ab. Es werden die Position und der Wert der Strömungsmaxima aufgenommen. Die Messreihe wird wieder für alle
möglichen Kopplungskapazitäten auf dem Schaltkasten wiederholt.