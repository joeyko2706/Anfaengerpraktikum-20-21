\section{Einleitung}
\label{sec:Einleitung}
Ziel des Versuches ist es gekoppelte Schwingkreise zu untersuchen. Obwohl im folgenden ein elektromaknetischer
Schwingkreis betrachtet wird, lassen sich die Erkenntnisse leicht auf ein mechanisches Analogon übertragen
(zum Beispiel ein gekoppeltes Schwingungssystem, bestehend aus 2 Fadenpendeln, die über eine elastische Feder miteinander verbunden sind 
\cite{Versuchsanleitung}). Der Grund, dass am elektrischen Schwingkreis Untersuchungen vorgenommen werden, ist dass die Amplitude und die Frequenz
einfacher und genauer bestimmt werden können.
Bei der Beobachtung des Schwingkreises wird auf die Energieverteilung der Systeme und auf den Einfluss
eines äußeren Erregers auf das schwingende System geachtet.

Die Erkenntnisse werden anschließend ausgewertet und mit der Theorie abgeglichen.