\section{Diskussion}
\label{sec:Diskussion}
Abschließend muss festgestellt werden, dass sich die Messwerte teilweise deutlich von den errechneten Theoriewerten unterscheiden. Dies ist zu großen Teilen darauf zurückzuführen, dass bei
dem im Versuch verwendeten Oszilloskop während des Versuches ein Wackelkontakt entdeckt wurde. Dieser äußerte sich in Form von plötzlichen Phasenverschiebungen, die ohne äußerliche
Einwirkung auf eines der Eingangssignale einwirkten. Dies machte die vorhergegangene Justierung der Kapazität des zweiten Schwingkreises teilweise zunichte und wirkte sich auf alle
folgenden Messungen aus.
\newline
Als Wert für die Summenfrequenz $ v^{+} $ wurde der Wert
\begin{equation*}
    v^{+} = 38.0 \si{\kilo\hertz}
\end{equation*}
gemessen, der errechnete Theoriewert beläuft sich auf
\begin{equation*}
    v^{+} = 35.32 \si{\kilo\hertz}.
\end{equation*}
Dies entspricht einer Abweichung von ungefähr 7.05\%.
\newline
Für die Differenzfrequenz entspricht der beste Messwert, der bei einer Kapazität von $C_{\text{K}} = 12 \si{\nano\farad}$ gemessen wurde
\begin{equation*}
    v^{-} = 49 \si{\kilo\hertz}.
\end{equation*}
Der zugehörige Theoriewert ist
\begin{equation*}
    v^{-} = 32.80 \si{\kilo\hertz}.
\end{equation*}
was einer Abweichung von etwa 49.39\% entspricht.
\newline
Der Messwert mit der größten Abweichung wurde bei einer Kapazität von $C_{\text{K}} = 1 \si{\nano\farad}$ gemessen und beträgt
\begin{equation*}
    v^{-} = 76 \si{\kilo\hertz}.
\end{equation*}
wobei sich der Theoriewert auf folgenden beläuft:
\begin{equation*}
    v^{-} = 47.52 \si{\kilo\hertz}.
\end{equation*}
Die Abweichung beträgt somit ungefähr 59.93\%.