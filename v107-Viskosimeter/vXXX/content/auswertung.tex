\section{Auswertung}
\label{sec:Auswertung}

\begin{table}[H]
  \centering
      \caption{Abmaße der kleinen Kugel.}
      \label{tab:klKugel}
      \begin{tabular}{c c}
      \toprule
      $m \:/\: \si{\gram} $ & $r \:/\: \si{\cm}$\\
      \midrule
      4,44 & 0,78 \\ 
       & 0,78 \\
       & 0,785 \\
       & 0,785 \\
       & 0,78 \\
      \bottomrule
  \end{tabular}
\end{table}

\begin{table}[H]
  \centering
      \caption{Abmaße der großen Kugel.}
      \label{tab:grKugel}
      \begin{tabular}{c c}
      \toprule
      $m \:/\: \si{\gram} $ & $r \:/\: \si{\cm}$\\
      \midrule
      4,91 & 0,795 \\ 
       & 0,795 \\
       & 0,795 \\
       & 0,79 \\
       & 0,795 \\
      \bottomrule
  \end{tabular}
\end{table}

\begin{table}[H]
  \centering
      \caption{Fallzeiten der kleinen Kugel bei Raumtemperatur ($\SI{19}{\celsius})$.}
      \label{tab:klKugRaum}
      \begin{tabular}{c c}
      \toprule
      Runter $t\:/\:$ s & Hoch $t\:/\:$ s\\
      \midrule
        12,87 & 13,13 \\
        12,79 & 13,00 \\
        12,42 & 12,89 \\
        12,66 & 13,02 \\
        12,93 & 12,88 \\
        12,68 & 12,88 \\
        12,80 & 12,69 \\
        12,68 & 13,01 \\
        12,94 & 12,95 \\
        12,29 & 12,94 \\
      \bottomrule
  \end{tabular}
\end{table}

\begin{table}[H]
  \centering
      \caption{Fallzeiten der großen Kugel bei Raumtemperatur ($\SI{19}{\celsius})$.}
      \label{tab:grKugRaum}
      \begin{tabular}{c c}
      \toprule
      Runter $t\:/\:$ s & Hoch $t\:/\:$ s\\
      \midrule
        39,34 & 42,70 \\
        41,82 & 42,02 \\
        42,68 & 41,29 \\
        42,20 & 41,28 \\
        42,16 & 42,38 \\
      \bottomrule
  \end{tabular}
\end{table}