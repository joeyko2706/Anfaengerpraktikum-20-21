\section{Durchführung}
\label{sec:Durchführung}
%Zu jedem Bild, jeder Tabelle, jeder Abbildung im allgemeinen einen Satz schreiben

\subsection{Wheatsone}
Der Versuc wird nach \autoref{fig:wheatstone} aufgebaut. Die Speisespannung wird an der Spannungsquelle eingestellt, 
darf aber, aufgrund technischer Begrenzung der Bauteile, $\SI{1}{\volt}$ nicht übersteigen. Es wird Wechselstrom 
verwendet. Die Brückenspannung wird mithilfe eines digitalen Oszilloskopes visualisiert. Das 
Widerstandsverhältnis $R_3/R_4$ wird durch ein Zehngang-Präzisionspotentiometer mit $\SI{1}{\kilo\ohm}$ Gesamtwiderstand,
so abgestimmt, dass die auf dem Oszilloskop angezeigte Spannung möglichst gering ist. Der Widerstand $R_2$ ist dabei fest, 
wird aber für eine Fehlermessung einmal variiert werden, sobald der unbekannte Widerstand $R_{\text{x}}$ einmal gemessen wurde.
Hochfrequente Störspannungen werden durch einen Tiefpass, der in die Schaltung mit eingebaut wird, weitgehend unterdrückt.

\subsection{Kapazitätsmessbrücke}
Der Versuch wird nach \autoref{fig:schaltungb} aufgebaut. Es werden nun zwei Potentiometer individuell variiert, sodass
Phase und Spannung auf dem Oszilloskop verschwinden. 

\subsection{Induktivitätsmessbrücke}
Der Versuch wird nach \autoref{fig:schaltungc} aufgebaut und die Messungen analog zur Messung der Kapazitätsmessbrücke durchgeführt.

\subsection{Maxwell-Brücke}
Der Versuch wird nach \autoref{fig:schaltungd} aufgebaut und die Spule ein zweites mal vermessen.

\subsection{Wien-Robinson-Brücke}
Der Versuch wird nach \autoref{fig:schaltunge} aufgebaut.
Im Bereich von $20$ bis $\SI{30 000}{\hertz}$ wird zunächst ein grobes Abbild der Brückenspannung abgemessen, bis
dann auf den Tiefpunkt in der Messung besondere Rücksicht genommen wird. Um de Tiefpunkt werden wietere, genauere Messung vorgenommen.