\section{Diskussion}
\label{sec:Diskussion}
Es fällt allgemein auf, dass die Fehler der Messwerte und die baubedingten Fehler oft stark auseinander liegen.

Die Messwerte der Wheatston'schen Messbrücke waren dabei die besten. Die Messwerte weichen um lediglich 0,3\% ab.

Bei der Kapazitätsmessbrücke hingegen konnte nur ein realistischer Messwert aufgenommmen werden. Dies lag daran, dass die Brückenspannung für kein Verhältnis $\frac{U_3}{U_4}$
ein Minimum hatte, sondern diese immer tiefer wurde, je größer $\frac{U_3}{U_4}$ wurde.

Bei der Induktivitätsmessbrücke weichen die Werte für den Widerstand um 7\% und die für die Induktivität um 49\% ab.

Auch bei der Maxwellbrücke weichen die Werte um ähnliche Anteile ab. Der Widerstand um 15\% und die Induktivität um 43\%

Auch der Klirrfaktor erscheint mit $k = 2,1466$ äußerst hoch.