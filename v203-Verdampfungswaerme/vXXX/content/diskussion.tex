\section{Diskussion}
\label{sec:Diskussion}
Zunächst muss erwähnt werden, dass die Messwerte von sämtlichen Messungen der ersten Messreihe mit Druck bis 1 Bar, bei weitem nicht den erwarteten Werten entsprechen. Zudem konnte
die Messung nicht bis zum angestrebten Druck von 1000 mBar durchgeführt werden. Dies ist der Fall, da die Mesapparatur sich ab einer Dampftemperatur von 84°C nicht mehr weiter
erhitzt hat, obwohl die Heizstärke schon bei einem scheinbaren Temperaturplateau von 45°C deutlich angehoben wurde. Zusätzlich wurde bei einem weiteren Temperaturstillstand bei 76°C
die Kühlung des Dampfes verringert, was die Temperatur jedoch nur kurzfristig erneut zum steigen brachte. Außerdem ist durch diesen Eingriff der Druck gesunken, was den Knick bei
den Messwerten in \autoref{fig:plot1} bei 77°C erklärt.