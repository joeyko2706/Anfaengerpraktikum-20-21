\section{Durchführung}
\label{sec:Durchführung}

\subsection{Allgemein}
Es werden insgesamt 3 Spulen untersucht. Eine lange Spule, 3 verschiedene Anordnungen eines Helmholtz-Spulenpaares und eine Toroidspule mit Eisenkern.
Bei der ersten beiden Spulen(anordnungen) wird ein konstanter Strom eingestellt und das erzeugte Magnetfeld, wird in Abhängigkeit zum Abstand gemessen.
Bei der Toroidspule wird der Spulenstrom variiert, um eine Hysteresekurve anzufertigen.

\subsection{Helmholtz-Spulenpaar}

Beim Helmholtz-Spulenpaar soll vor allem die physikalische Besonderheit untersucht werden, dass das Magnetfeld in ihrem gemeinsamen Mittelpunkt auf der
rotationssymmetrischen Achse näherungsweise homogen ist. 
Für diese Besonderheit muss der Abstand der beiden Spulen zueinander dem Radius der Spulen entsprechen.

Die beiden Spulen sind auf einer Schiene in einer Halterung befestigt worden, auf der eine Spule fest und die andere frei beweglich ist. 
Die Halterung besitzt eine weitere Schiene, die es ermöglicht eine Hall-Sonde in das Magnetfeld der Spule einzulassen und somit die
Magnetflussdichte des Spulenpaares zu messen. Da beide Schienen eine eigenes, festes Lineal besitzen, kann man den Spulen- und den Sondenabstand
genau einstellen.

Es werden 3 verschiedene Spulenabstände eingestellt und deren magnetische Flussdichte in Abhängigkeit der Position zu den Spulen gemessen.

Für diesen Aufbau wird eine transversale Hall-Sonde verwendet. Es wird darauf getachtet, dass die Sonde orthogonal zu der Symmetrieachse beider
Spulen platziert wird. Die beiden Spulen werden in Reihe geschaltet, wobei darauf geachtet werden muss, dass der Stom langsam von Null hochgedreht
wird. Dabei darf der maximal zulässige Spulenstrom nicht überschritten werden. Bevor jedoch der Strom hochgedreht wird, schaltet man das Messgerät
für die magnetische Feldstärke ein und setzt es durch drücken der Taste \textbf{Null} auf Null.

Über den Knopf \textbf{Range} stellt man ein, welche Stellen des Messwertes angezeigt werden sollen.

Anschließend werden die Messungen, wie oben beschrieben durchgeführt.

\subsection{lange Spule}

Bei der Messung der magnetischen Flussdichte einer langen Spule wird eine longitudinale Hall-Sonde verwendet.
Die Spule wird auf ein festes Lineal gestellt, an dessen Ende sich ein Stativ befindet, in dessen Stellschraube eine Hall-Sonde ist.
Die Hall-Sonde ist möglichst auf der Symmetrieachse der langen Spule anzubringen. 
Die lange Spule wird an den Strom angeschlossen. Dabei ist darauf zu achten, dass der Strom anfangs auf Null steht und dann langsam hochgedreht
wird. Der maximale Spulenstrom darf nicht überschritten werden.

Es werden nun sowohl in der Spule, als auch außerhalb, Messungen vorgenommen.

\subsection{Toroidspule mit Eisenkern}
Die Toroidspule hat einen für die Messungen vorgesehenen Luftspalt, in dem die Messungen vorgenommen werden. Die Spule wird an den Strom
angeschlossen und eine transversale Hall-Sonde in den Luftspalt eingelassen. Es ist darauf zu achten, dass der maximale Spulenstrom nicht überschritten
wird.

Die Messungen werden in gleichmäßigen Inkrementen des Stroms vollzogen. Da bei der Spule ein Maximalwert von 10 Ampere nicht überschritten werden
darf, ist es sinnvoll die Abstände auf 1 Ampere zu wählen. Die größe der Inkremente ist im allgemeinen jedoch frei zu wählen, solange sie gleichmäßig
und sinnvoll sind.

Ist ein Spulenstrom von +10 Ampere erreicht, senkt man ihn in 1-Ampere-Schritten wieder auf Null. Danach polt man die Toroidspule um und
erhöht den Spulenstrom auf -10 Ampere. Ist dieser Wert erreicht, wird wieder auf 0 und danach auf +10 Ampere ehröht.
Es zeichnet sich eine Hysteresekurve ab.